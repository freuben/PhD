\hypertarget{chapter5}{}
\chapter{Appropiation and (Post)-postproduction}

\hypertarget{appropriation}{}
\section {Appropriation Art}

\begin{quote}
The contemporary era constantly proclaims itself as post-ideological, but this denial of ideology only provides the ultimate proof that we are more than ever embedded in ideology. Ideology is always a field of struggle---among other things, the struggle for appropriating past traditions.\footnote{\hyperlink{zizektragedy}{\v{Z}i\v{z}ek (2009)}, `It's Ideology, Stupid!', p. 37.}
\end{quote}
Start on appropriation and past traditions...

%\begin{quote}
%Consumption is simultaneously also production, just as in nature the production of a plant involves the consumption of elemental forces and chemical material\footnote{Karl Marx}
%\end{quote}

\begin{quote}
Starting with the language imposed upon us (the \emph{system} of production), we construct our own sentences (\emph{acts} of everyday life), thereby reappropriating for ourselves, through these clandestine microbricolages, the last word in the productive chain. Production thus becomes a lexicon of a practice, which is to say, the intermediary material from which new utterances can be articulated, instead of representing the end result of anything. What matters is what we make of the elements placed at our disposal. . . . By listening to music or reading a book, we produce new material, we become producers. And each day we benefit from more ways in which to organize this production: remote controls, VCRs, computers, MP3s, tools that allow us to select, reconstruct, and edit. Postproduction artists are agents of this evolution, the specialized workers of cultural reappropriation.\footnote{\hyperlink{postproduction}{Bourriaud (2005)}, pp. 24-25.}
\end{quote}

\subsection{Appropriation and Postproduction in the Digital Age} 

\begin{quote}
Throughout the eighties, the democratization of computers and the appearance of sampling allowed for the emergence of a new cultural configuration, whose figures are the programmer and DJ. The remixer has become more important than the instrumentalist, the rave more exciting than the concert hall. The supremacy of cultures of appropriation and the reprocessing of forms calls for an ethics: to paraphrase Philippe Thomas, artworks belong to everyone. Contemporary art tends to abolish the ownership of forms, or in any case to shake up the old jurisprudence. Are we heading toward a culture that would do away with copyright in favor of a policy allowing free access to works, a sort of blueprint for a communism of forms?\footnote{Ibid. p. 35.}
\end{quote}

\subsection{The liberal-comunists: Open Source, etc.} 

critisisms...

There is no music by John Oswald on the net free to download. Hypocrisy from the appropriator? Or does he fall into the logic of late-capitalism - �no communism of forms�? �I plunder but don�t plunder me. Or, at least not for free��? 

I propose an attitude towards music appropriation similar to that of hacker communities and the open source initiative. Not with the purpose of suggesting a communist utopia, but of being consequent with my creative process. By giving away my music, recorded sounds and experiments, code, etc, through the net, I will hopefully instigate others to do so as well. If this attitude is followed, it could promote the organization of music cyber communities that would plunder, engage with and promote each other, hopefully producing more subversive types of music.

We are far from the Bourriaud�s utopia. The only people how have access to (artistic) shareware are commoditized people, mostly in western countries. Isn�t the DJ approach towards plunderphonics one that appropriates to make more profit and diminish costs only to thereafter feed back their product into the music industry system?

\subsection{appropriation, ideology and the use of references}

\begin{quote}
While some start up a prolonged lamentation for the lost image, others reopen their albums to rediscover the pure enchantment of images- that is, the alterity of the \emph{was}, between the pleasure of pure presence and the bit of the absolute Other.
\end{quote}
\begin{quote}
Evidence of exhibitions devoter to `images', but also the dialectic that affects each type of image and mixes its legitimations and powers with those of the other tow.
\end{quote}

\section{Musical Appropriation}

I will continue by examining different strategies and practices used in my work that use technology as means to appropriate, derive from and transform existing music by other musicians. It is only logical, considering that music is not an object but a complex set of actions, productions, perceptions and thoughts,\footnote{See pp 13-45 for a discussion regarding my preference of the notion of a \emph{musical result} versus the more widely use concept of \emph{musical work}.} that the act of appropriation of existing music can manifest itself in many different ways and take lots of unexpected guises. Therefore, I will propose that the appropriation of existing music \emph{does not} refer exclusively to `borrowing' or `stealing' from \emph{musical works} by other composers but to . . . . Moreover, when dealing with appropriation, I will claim that there are certain fundamental questions that both music creators and listeners should ask themselves. According to David Mezter, Stockhausen (while referring to \emph{Hymnen}) emphasized the importance of asking the questions of ``what'' and ``how''  regarding the practice of `borrowing' or `quoting' from other music.
\begin{quote}
According to him \emph{[Stockhausen]}, the practice involves a rich exchange between the ``what'' and the ``how'', that is, the gesture has us hear ``what'' music has been borrowed and ``how'' it has been changed. The more familiar and obvious the ``what'', the more we are drawn into the ``how'', and the more captivating the ``how'', the more we can appreciate anew the ``what''. It is the ways in which quotation handles the ``what'' and the ``how'' that make it so effective a  cultural agent.\footnote{\hyperlink{metzer}{Metzer (2003)}, p. 6.} 
\end{quote}
I agree with Stockhausen's claim because . . . 
Nevertheless, I would also add : 
the difference between ``what'' and ``who''
also ``from where''. 
but most importantly ``why'',  
Why = motivations.
The motivations regarding musical appropriation can be very varied and also reflect ideological positions that in many cases reflect more the beliefs and feelings of the appropriator that the appropriated. Therefore, I will attempt to explain my viewpoint regarding the motivations and ways in which I use other music within my own work. In doing so, I will also examine other composers work that deals with musical appropriation in ways that I consider valid, interesting and intriguing.

Technology

Will do so by examining other composers work dealing with this issues... that I find valid, interesting, intriguing, stimulating(?)...  

\subsection{Musica Derivata and Plunderphonics}

``A good composer does not imitate; he steals''       I. Stravinsky

Musica Derivata:

``music that is compositionally based on other music'' (K. Barlow) 


\subsection{Questions Regarding Appropriation}

What? 

Code, compositional tecniques, what piece of music? 
Do we plunder from the ``flea market or (the) airport shopping mall''? (N. Bourriaud). From the top 20 list - J. Oswald approach-, or from the hidden CDs at the back of the music store?

Who?

Music Industry? Pop/commercial? Historical (dead composers)? Music from different cultures? 

Appropriation of the Other. What relationship do we want to establish with the Other? Impersonal like the 1st/3rd World relationships?

Liberal multiculturalists approach? ``Other deprived of its Otherness (the idealized Other who dances fascinating dances and has an ecologically sound holistic approach to reality, while features like wife beating remain out of sight�)?'' (Slavoj \v{Z}i\v{z}ek, 2003)

Why?

For the meaning of the cultural object you are appropriating? For it�s symbolism? To suggest a metaphor?

For it�s use? ``Don�t look for the meaning, look for the use'' - L. Wittgenstein - for example for the sonic qualities of the appropriation (intonation, groove, etc.)

How? �

\subsection{Types of Musical Appropriation}

\subsubsection{Copyrights Violation}
 
\subsubsection{Scores}


The first strategy considered is Clarence Barlow's concept of \emph{Musica Derivata}, which refers to the idea of transforming existing music with Computer Aided Composition (CAC) tools to create ``music that is compositionally based on other music''\footnote{\hyperlink{barlow}{Barlow (2000)}.} This approach seems to take as a starting point mostly notated material (but in some occasions spectral information from recordings) from music by other composers. 

Midi

\subsubsection{me}

\subsubsection{Recordings}

	
\subsubsection{Plunderphonics}

Plunderphonics:

John Owald, 1985. ``Plunderphonics, or Audio Piracy as Compositional Prerogative''

Use of audio samples as a technique for composition. 

Different from Musica Derivata in that it appropriates the recording of the original musical source. Information from recording (tibre, rhythm, performace practice, etc) is plundered from the original source to create a new composition.

``As a listener my own preference is the option to experiment. My listening system has a mixer instead of a receiver, an infinitely variable speed turntable, filters, reverse capability, and a pair of ears. An active listener might speed up a piece of music in order to perceive more clearly it�s macrostructure, or slow it down to hear articulation and detail more precisely''.\footnote{\hyperlink{oswald}{Oswald (1985)}.}

\subsubsection{Sound Transformations} 					


``With the power of the computer, we can transform sounds in such radical ways that we can no longer assert that the goal sound is related to the source sound merely because we have derived one from the other''. (T. Wishart)

In my work, sound transformations are used for the transformation of existing music. 

Why transformation of musical sources? Because they may carry complex cultural symbolism. 

The amount of processing can affect our ability to recognize the source sound or musical sample. Therefore, there is a wide palette of derivative music available to us: from the radically processed � less recognizable source � more `abstract' extreme; to the less processed � more recognizable source � more `referential'  and quotation type music.

Performance practice and other sonic characteristics of many original musical sources is lost in the transcription to a fully notated score for ensembles of western classically trained musicians. Many aspects of sound production (intonation, groove, spectral characteristics of instruments/voices, etc) is lost via this process.

Process of derivation and sound transformation is not directly apparent to the audience. The act of appropriation is not transparent.


\subsubsection{Spectral Information} 

\subsubsection{To generate sc}

\subsubsection{Computer Code}

Max patches, Computer Code.

\subsubsection{Real Performances} 


\subsubsection{Real-Time Plunderphonics}

Appropriation of audio signals from live music performances as material for a new composition

Creates a cognitive dissonance between audio and visuals.

The amount of processing of the audio signals is visible. The more processed the performances are, the more contrasting they will look in relationship with what is heard through speakers.

In contrast to acousmatic tradition, Real-Time Plunderphonics makes the process of appropriation transparent to the audience through the cognitive association between audio and visuals.

Changes relationship with the appropriated Other: The performer becomes an accomplice in the process of appropriation (or themselves). 

Deals with the problematic of the lack of visual clues and theatrical elements in electronic music performance by introducing a dynamic group of live performers and an interesting and unusual visual scenario.  

\subsubsection{Some ideas of how to plunder}

Get to know what and who you are plundering and figure why your are doing so before you decide how to plunder.(Know your performers, their music and why you want to work with them)

Appropriate and plunder yourself. 

Plundering not as central purpose of the creative process, but rather a tool for creating new idiosyncratic audio/visual result. 

Use ``from raw to cooked'' (L\'{e}vi-Strauss) techniques to create a narrative that navigates, in literary terms, between the �real� (actual performance) and the `surreal' (extreme processed audio).

Combinations of Real-Time Plunderphonics, (Real-Time) Musica Derivata and Sound Transformations

Use plunderphones as data: reprogram, not just remix.

Micro and macro plundering.

Use also Non Real-Time tools (Scores, Samples, etc.) if suitable. 

Using plunderphones as data

An example: Use FFT data of your plunderphone to trigger samples of recorded instruments.

\subsubsection{Micro and Macro Plundering}

Microplunderphonics

Plundering just microelements of sound. Not the whole spectrum of the original sound file. 

Generate noise with your plunderphones and use it instead of white noise for sound synthesis


Macroplundering

Appropriate a composition�s form. Use the structure as blueprint for a new composition. 

Use variables of the appropriated piece (pitch, dynamics, etc.) as control structures for new output.

\label{ch:approp}