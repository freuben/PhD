\hypertarget{chapter5}{}
\chapter{Appropriation as Strategy}

This chapter examines musical practices that take an explicit and formalized approach to the use of appropriation as a conceptual, performative and compositional strategy.
Not a survey...

\section{Appropriation, Ideology and the Past}

Appropriation is certainly nothing new to music. Musicians have used, borrowed and stolen from the work of others for centuries, either explicitly or without deliberation. The very notion of what it means to make music is implicitly related to the act of appropriating existing sounds, structures, actions and thoughts. As with any other type of artistic production, music depends on already existing materials and actions to produce something new. In the case of musical appropriation, the existing materials used not only refer to the physical materials or objects utilized to create music (the instrument, the strings of the piano, the stage, the musician's own body, \emph{etc.}), but also to its content (already existing sounds, rhythms, structures, gestures, timbres, \emph{etc.}). Nevertheless, musical appropriation is not limited only to the act by which musicians choose their materials to kick-start the creative process. The notion of musical appropriation also includes the process of determining the system of musical production itself, which is based on a priori forms of knowledge, abstractions, deductions and subjective processes that as a whole constitute a set of creative frameworks and stratergies of production. Put briefly, not only does the musician appropriate existing materials to work with, but the method of working itself. Furthermore, the methods by which the musician works may not only be appropriated from within the traditional conventions of music-making but also from the knowledge gained from other forms of thought and production that come from disciplines outside music. In addition, the act of appropriation is not always deliberate. One, as an musician paradoxically might not be aware of all of the sources that one drives inspiration from---one might just causally absorb and appropriate from ones own experience by \emph{living} and dwelling within ones surrounding culture. Consequently, for as long as human beings have been producing music, there has been strategies of musical appropriation. The notions of creativity and originality in music are therefore not independent from that of appropriation and a debate about music is only futile if one starts from the premise that musicians either appropriate or not. A much more fruitful discussion would stem from the presupposition that all musicians appropriate---they all use existing methods and materials, they borrow and steal from each other and from other existing cultural and historical modes of production---and the questions that are more relevant are what one as a musician appropriates, who and where one appropriates from, how and why one appropriates and what one accomplishes through appropriation. What the musician does with the appropriated methods and materials and with what purpose is therefore much more relevant to the notions of creativity and originality than questions of authorship, copyright or legitimacy concerning an individual been responsible for the action of making music. 

What's more, whether deliberate or not, within the act of appropriation lies in essence the musician's relationship to the past. This relationship may be established through the materials and methods the musician appropriates as well as how they have been appropriated. For instance, a clear relationship could be established with the work of an individual composer from the past---either systematically or intuitively---by imitating, copying, interpreting or modifying materials and methods from his/her body of work. Therefore, the relationship between the musician and the past can be one that is initiated by a personal interest (which might be objective or subjective) in the musical practices and output of a specific historical figure. The process of appropriation from the work of historical figures that a musician admires or relates to is, I will claim, how many musicians start to form their (musical) identity. However, I believe that one as a musician---and therefore a cultural agent---should not only rely on relationships with the past based on personal and subjective impressions of an individual's musical practices and output without considering their work within a wider cultural and historical context. Therefore, when one is appropriating from another composer's work, one is not merely referring to an isolated body of work but accessing a complex set of relationships in connection to how the composer relates to his/her surrounding culture and what is the significance of his/her work in a wider historical and cultural context. That is to say, even when we appropriate from just one musical source from the past, we are in fact making reference to a panoply of historical and cultural symbols that today might have a complex order of significance. Nevertheless, 
the relationship between the musician and the past is not only reflected by who creates the materials and methods of music-making and what those are, but also on how the musician might appropriate them. Thus, one may deduce more (depending on the nature of the \emph{musical result}) about the musicians's relationship with the past through the way in which he/she appropriates an existing piece of music, than who possesses the original authorship of the composition and what cultural/historical significance it might have. 
Similarly, a musician's relationship with the past might be contingent to how he/she uses different types of performance practice for his/her own music-making, rather than what these practices are or what they signify.

Additionally, within the act of appropriation lies a deep relationship between the musician who appropriates and the appropriated Other. How one as appropriator selects the material (whether it is a specific score by a composer from western music history or an `ethnic' instrument from a non-western culture) and acts towards it, reveals a type of relationship established between oneself and the Other. These types of relationships disclose (either objectively or subjectively) an attitude towards the Other that displays ones position in reference to the struggle of cultural appropriation. This attitude however might not be the one intended by the appropriator. That is to say, even though the intention of the musician who appropriates from the Other is to convey a specific type of attitude, in reality it might show a rather different one. That is the case today for instance, with a common attitude in western countries towards non-western cultures that is reflected through musical appropriation and can be associated with the notion of \emph{multiculturalism}.\footnote{See \hyperlink{multiculturalism}{pp. 43-44} for a wider discussion on the concept of \emph{multiculturalism} and  Slavoj \v{Z}i\v{z}ek's interpretation of what it means in actuality.} Even though the initial intent of a composer who incorporates instruments and performers from different cultures is to establish an attitude that reflects equality and tolerance, by treating the Other with `respect' and assuming a superfluous position of `openness', in reality what this type of appropriation might exhibit is a patronizing and condescending distance that situates the western composer in a position of superiority in the struggle that is an intercultural exchange. Therefore, determining the appropriator's attitude towards the appropriated Other requires a critical evaluation and reflection of the complex processes that take place during musical appropriation. Put briefly, one should not infer that the initial attitude one hopes to assume regarding the Other (without putting it to through rigorous scrutiny) will be reflected in the type of relationship that is eventually established. Furthermore, one cannot simply rely on `first impressions' concerning ones preconceived notions of the Other or the conviction that one will decipher his/her otherness through a process of acculturation. I believe that one can only take a sensible stance towards appropriation when one accepts the fundamental impossibility of fully understanding the Other. Thus, a `constructive' exchange regarding musical appropriation only starts when one agrees to disagree with the Other in some aspects of his/her musical traditions. Moreover, one should not forget that the act of appropriation in-itself implies a form of violence and confrontation between different cultures, traditions and social strata.

The way in which one as a musician appropriates also reflects a deep connection to ones ideology and the way one relates to tradition. Even though music rarely has a \emph{raison d'\^{e}tre} that is simply ideological, it nevertheless reflects ideology, even without been entirely aware of it. Additionally, because of music's non-conceptual and non-objective nature, ideology will not be revealed within it as clearly and directly as it would for instance through language. However, ideological positions maybe traced through what happens within music, including how, where and by whom it is created, presented, perceived and consumed. Music's subjective characteristics nevertheless make the methodology of examining ideology in music, an analytical and interpretative process. Therefore, one can only demonstrate ideological positions in music through critical reflection. That been said, I believe that by analyzing the way in which a musician appropriates from tradition as well as which traditions he/she appropriates (considering his/her own background), we might widen our understanding about his/her ideology. Also, through this process we can also grasp the wider ideological context in which the music dwells---in other words, we should not only take into consideration the ideological position of the musicians themselves but think about the bigger cultural picture, which includes the prevailing ideologies of all the people involved in the creation and reception of the music (the institutions involved in a musical performance, the type of audience, the type of venue, the way in which the music is disseminated and consumed, \emph{etc}.). Consequently, the link between ideology and musical appropriation at once makes music a discipline embedded within ideology and displays a significant link between musical and cultural appropriation. Slavoj \v{Z}i\v{z}ek has argued that cultural appropriation and ideology imply an act of violence and friction that is created through the appropriation of past traditions. According to \v{Z}i\v{z}ek, in today's global society, ideology is more relevant than ever before, considering the violence implied in the act of cultural appropriation.
\begin{quote}
The contemporary era constantly proclaims itself as post-ideological, but this denial of ideology only provides the ultimate proof that we are more than ever embedded in ideology. Ideology is always a field of struggle---among other things, the struggle for appropriating past traditions.\footnote{\hyperlink{zizektragedy}{\v{Z}i\v{z}ek (2009)}, `It's Ideology, Stupid!', p. 37.}
\end{quote}
Given the relationship that exists between musical and cultural appropriation, musical appropriation may be deliberately used as an ideological tool and as a strategy to convey thoughts, opinions and feelings associated with the struggle of cultural appropriation. 

Past traditions . . .

Moreover, through the intentional and specific use of artistic appropriation as a an act of violence, the artists may denounce 

In today's globalized society . . . Given the fact that 

Appropriated material reflects ideology . . . \v{Z}i\v{z}ek/Adorno but. . . . The artist can present their own view of these references by rearranging them modifying them. The appropriation artist doesn't necessarily adheres to the ideology of the appropriated material, but reflects it by the use of the appropriation - how are they presented, modified, etc?  

How one relates to the Other. Other cultures too... (Past) Traditions

Adorno ---``music is not ideology'' . .  .  .



\section{Appropriation Art}

Appropriation artists deliberatly/consciously use appropriation - formalization of appropriation. 

\subsection {Readymades and D\'etournement}

- Duchamp \\
- Reciprocal Readymade.\\

\begin{quote}
In 1913 I had the happy idea to fasten a bicycle wheel to a kitchen stool and watch it turn. A few months later I bought a cheap reproduction of a winter evening landscape, which I called `Pharmacy' after adding two small dots, one red and one yellow, in the horizon. In New York in 1915 I bought at a hardware store a snow shovel on which I wrote `in advance of the broken arm'. It was around that time that the word `readymade' came to mind to designate this form of manifestation. . . . At another time---wanting to expose the basic antimony between art and readymades---I imagined a `reciprocal readymade': use a Rembrant as ironing board!\footnote{\hyperlink{duchamp}{Duchamp (2009)}, p. 40.}
\end{quote}

- D\'etournement \\
- Guy Debord \\

\subsection {Postproduction}

\begin{quote}
Starting with the language imposed upon us (the \emph{system} of production), we construct our own sentences (\emph{acts} of everyday life), thereby reappropriating for ourselves, through these clandestine microbricolages, the last word in the productive chain. Production thus becomes a lexicon of a practice, which is to say, the intermediary material from which new utterances can be articulated, instead of representing the end result of anything. What matters is what we make of the elements placed at our disposal. . . . By listening to music or reading a book, we produce new material, we become producers. And each day we benefit from more ways in which to organize this production: remote controls, VCRs, computers, MP3s, tools that allow us to select, reconstruct, and edit. Postproduction artists are agents of this evolution, the specialized workers of cultural reappropriation.\footnote{\hyperlink{postproduction}{Bourriaud (2005)}, pp. 24-25.}
\end{quote}

\subsection{Appropriation in the Digital Age}

\begin{quote}
Throughout the eighties, the democratization of computers and the appearance of sampling allowed for the emergence of a new cultural configuration, whose figures are the programmer and DJ. The remixer has become more important than the instrumentalist, the rave more exciting than the concert hall. The supremacy of cultures of appropriation and the reprocessing of forms calls for an ethics: to paraphrase Philippe Thomas, artworks belong to everyone. Contemporary art tends to abolish the ownership of forms, or in any case to shake up the old jurisprudence. Are we heading toward a culture that would do away with copyright in favor of a policy allowing free access to works, a sort of blueprint for a communism of forms?\footnote{Ibid. p. 35.}
\end{quote}

\subsubsection {Copying and Sharing: Social (Re)appropriation}

- Hackers \\
- Open Source \\

\subsubsection{Liberal-communists} 

- Zizek, Bill Gates model \\

\subsection {Postmodernism and Appropriation}

Postmodernism - Institutionalization of Modernism.

Relationship between Postmodern art and appropriation.

Criticisms of Postmodern art practices?

\subsubsection{Considerations} 

What? 

Code, compositional tecniques, what piece of music? 
Do we plunder from the ``flea market or (the) airport shopping mall''? (N. Bourriaud). From the top 20 list - J. Oswald approach-, or from the hidden CDs at the back of the music store?
\\
Who?

Music Industry? Pop/commercial? Historical (dead composers)? Music from different cultures? 

Multiculturalism, Globalization., etc.

Appropriation of the Other. What relationship do we want to establish with the Other? Impersonal like the 1st/3rd World relationships?

Liberal multiculturalists approach? ``Other deprived of its Otherness (the idealized Other who dances fascinating dances and has an ecologically sound holistic approach to reality, while features like wife beating remain out of sight�)?'' (Slavoj \v{Z}i\v{z}ek, 2003)
\\
Why?

For the meaning of the cultural object you are appropriating? For it�s symbolism? To suggest a metaphor?

For it�s use? ``Don�t look for the meaning, look for the use'' - L. Wittgenstein - for example for the sonic qualities of the appropriation (intonation, groove, etc.)
\\
How?

\section{Musical Appropriation}

This is not a survey. For that ``Quotation and Cultural Meaning in Twentieth-Century Music.''\footnote{See \hyperlink{metzer}{Metzer (2003)}.}

\subsection{Postmodern Music}

Start with Berio--- transcriptions, etc. Then to more recent trends...

\subsection{Musica Derivata}

``music that is compositionally based on other music'' (K. Barlow) 

The first strategy considered is Clarence Barlow's concept of \emph{Musica Derivata}, which refers to the idea of transforming existing music with Computer Aided Composition (CAC) tools to create ``music that is compositionally based on other music''\footnote{\hyperlink{barlow}{Barlow (2000)}.} This approach seems to take as a starting point mostly notated material (but in some occasions spectral information from recordings) from music by other composers. 

Appropriation of Score Info: Performance practice and other sonic characteristics of many original musical sources is lost in the transcription to a fully notated score for ensembles of western classically trained musicians. Many aspects of sound production (intonation, groove, spectral characteristics of instruments/voices, etc) is lost via this process.

\subsubsection{Spectral Information} 

\subsection{Plunderphonics}

a little history... The Gramophone (ideas by Moholy-Magy\footnote{See \hyperlink{moholy}{Moholy-Nagy (2006)}}), John Cage, Stockhausen, Oswald, Negativeland etc.

John Oswald and Chris Cutler's articles.

\begin{quote}
As a listener my own preference is the option to experiment. My listening system has a mixer instead of a receiver, an infinitely variable speed turntable, filters, reverse capability, and a pair of ears. An active listener might speed up a piece of music in order to perceive more clearly it�s macrostructure, or slow it down to hear articulation and detail more precisely.\footnote{\hyperlink{oswald}{Oswald (1985)}.}
\end{quote}

\subsubsection{Copyrights}

\subsubsection{Sampling Culture}

\subsection{Other Ideas on Musical Appropriation}

\subsubsection{Sound Transformations} 					

``With the power of the computer, we can transform sounds in such radical ways that we can no longer assert that the goal sound is related to the source sound merely because we have derived one from the other''. (T. Wishart)

Palette: Recognizable (quotation) - to non-recognizable (``abstract'').

The amount of processing can affect our ability to recognize the source sound or musical sample. Therefore, there is a wide palette of derivative music available to us: from the radically processed � less recognizable source � more `abstract' extreme; to the less processed � more recognizable source � more `referential'  and quotation type music.

\subsubsection{Micro and Macro Plundering}

Microplunderphonics

Plundering just microelements of sound. Not the whole spectrum of the original sound file. 

Generate noise with your plunderphones and use it instead of white noise for sound synthesis

Macroplundering

Appropriate a composition�s form. Use the structure as blueprint for a new composition. 

Use variables of the appropriated piece (pitch, dynamics, etc.) as control structures for new output.

\subsubsection{Using Data and Mapping }

Using plunderphones as data

An example: Use FFT data of your plunderphone to trigger samples of recorded instruments.

\subsubsection{Sharing Code}

Max patches, Computer Code.

\subsubsection{Real-Time Plunderphonics}

Real-time possibilities of appropriation. Appropriation of Live Performances.

Appropriation of audio signals from live music performances as material for a new composition

Creates a cognitive dissonance between audio and visuals.

The amount of processing of the audio signals is visible. The more processed the performances are, the more contrasting they will look in relationship with what is heard through speakers.

In contrast to acousmatic tradition, Real-Time Plunderphonics makes the process of appropriation transparent to the audience through the cognitive association between audio and visuals.

Changes relationship with the appropriated Other: The performer becomes an accomplice in the process of appropriation (or themselves). 

\label{ch:approp}