\hypertarget{chapter4}{}
\chapter{Strategies and Practices}

This chapter aims at examining the musical concepts, strategies and practices that were elaborated during the period of research, taking in consideration the aesthetic and historical background developed in Chapter 2 and conceptual framework and motivation put forward in Chapter3. 
%I will start the discussion by describing different approaches that musicians have taken recently where music by other musicians (composers and performers)  is appropriated, derived from and transformed with the aid of technology.

\section{Technology and Innovation}

\begin{quote}
In other words, if in the past---even the distant past---music was often the testing bench and the stimulus for scientific research, and thus music tended to draw scientific knowledge to it, in more recent years you get the impression that it's now science that draws music to it and takes possession of it. Indeed, you often get the impression that a scientific creativity applicable to music has substituted itself for musical creativity, and that musical thought has regressed to the level of the (invariably squalid) opinions that an electronic engineer from Bell Telephone or a Stanford ``software man'' may have about music. . . . Thus many of the more sensitive musicians quickly realized that it was as easy as it was superfluous to produce new sounds that were not the product of musical thought, just as it's easy nowadays to develop and `improve'' the technologies of electronics music when there are devoid of any real and profound \emph{raison d'\^{e}tre}.\footnote{Luciano Berio, \emph{Two Interviews with Rossana Dalmonte and B\'{a}lint Andr\'{a}s Varga},  Ed. and Trans. David Osmond-Smith, London: Marion Bowars, 1985, pp. 121,122.} 
\end{quote}

\begin{quote}
It was recognized, for example, that the spectacle of a public gathered together to listen to loudspeakers was not a particularly cheerful one, and that, yet again, the experience of public musical listening was made up of may different conventions, and was rooted in many different aspects of social and cultural life: it was not made up merely of a piece, a musical object to listen to, even if it proposed ``new sounds''. By its very nature, a piece of music by itself cannot easily transform listening conventions and socio-musical relations in general.\footnote{Ibid., pp. 122,123.}
\end{quote}


\section{Musical Appropriation through Technology}

I will continue by examining different strategies and practices used in my work that use technology as means to appropriate, derive from and transform existing music by other musicians. It is only logical, considering that music is not an object but a complex set of actions, productions, perceptions and thoughts,\footnote{See pp 13-45 for a discussion regarding my preference of the notion of a \emph{musical result} versus the more widely use concept of \emph{musical work}.} that the act of appropriation of existing music can manifest itself in many different ways and take lots of unexpected guises. Therefore, I will propose that the appropriation of existing music \emph{does not} refer exclusively to `borrowing' or `stealing' from \emph{musical works} by other composers but to . . . . Moreover, when dealing with appropriation, I will claim that there are certain fundamental questions that both music creators and listeners should ask themselves. According to David Mezter, Stockhausen (while referring to \emph{Hymnen}) emphasized the importance of asking the questions of ``what'' and ``how''  regarding the practice of `borrowing' or `quoting' from other music.
\begin{quote}
According to him \emph{[Stockhausen]}, the practice involves a rich exchange between the ``what'' and the ``how,'' that is, the gesture has us hear ``what'' music has been borrowed and ``how'' it has been changed. The more familiar and obvious the ``what,'' the more we are drawn into the ``how,'' and the more captivating the ``how,'' the more we can appreciate anew the ``what.'' It is the ways in which quotation handles the ``what'' and the ``how'' that make it so effective a  cultural agent.\footnote{David Metzer, \emph{Quotation and Cultural Meaning in Twentieth-Centure Music}, Cambridge: Cambridge University Press, 2003, p. 6.} 
\end{quote}
I agree with Stockhausen's claim because . . . 
Nevertheless, I would also add : 
the difference between ``what'' and ``who''
also ``from where''. 
but most importantly ``why,''  
Why = motivations.
The motivations regarding musical appropriation can be very varied and also reflect ideological positions that in many cases reflect more the beliefs and feelings of the appropriator that the appropriated. Therefore, I will attempt to explain my viewpoint regarding the motivations and ways in which I use other music within my own work. In doing so, I will also examine other composers work that deals with musical appropriation in ways that I consider valid, interesting and intriguing.

Technology

Will do so by examining other composers work dealing with this issues... that I find valid, interesting, intriguing, stimulating(?)...  

\subsubsection{Copyrights Violation}
 
\subsection{Scores}


The first strategy considered is Clarence Barlow's concept of \emph{Musica Derivata}, which refers to the idea of transforming existing music with Computer Aided Composition (CAC) tools to create ``music that is compositionally based on other music''\footnote{Clarence Barlow, \emph{Musica Derivata} [CD], hat[now]ART 126, Hat Hut Records Basle, 2000.} This approach seems to take as a starting point mostly notated material (but in some occasions spectral information from recordings) from music by other composers. 

MIDI

\subsubsection{me}

\subsection{Recordings}

	
\subsubsection{Plunderphonics}

Plunderphonics:

John Owald, 1985. ``Plunderphonics, or Audio Piracy as Compositional Prerogative''

Use of audio samples as a technique for composition. 

Different from Musica Derivata in that it appropriates the recording of the original musical source. Information from recording (tibre, rhythm, performace practice, etc) is plundered from the original source to create a new composition.

``As a listener my own preference is the option to experiment. My listening system has a mixer instead of a receiver, an infinitely variable speed turntable, filters, reverse capability, and a pair of ears. An active listener might speed up a piece of music in order to perceive more clearly it�s macrostructure, or slow it down to hear articulation and detail more precisely.''\footnote{John Oswald, ``Plunderphonics, or Audio Piracy as a Compositional Prerogative,'' in \emph{Wired Society Electro-Acoustic Conference}, Toronto, 1985. URL: \href{http://www.plunderphonics.com/xhtml/xplunder.html}{http://www.plunderphonics.com/xhtml/xplunder.html}.}



\subsubsection{Sound Transformations} 					


``With the power of the computer, we can transform sounds in such radical ways that we can no longer assert that the goal sound is related to the source sound merely because we have derived one from the other.'' (T. Wishart)

In my work, sound transformations are used for the transformation of existing music. 

Why transformation of musical sources? Because they may carry complex cultural symbolism. 

The amount of processing can affect our ability to recognize the source sound or musical sample. Therefore, there is a wide palette of derivative music available to us: from the radically processed � less recognizable source � more `abstract' extreme; to the less processed � more recognizable source � more `referential'  and quotation type music.

Performance practice and other sonic characteristics of many original musical sources is lost in the transcription to a fully notated score for ensembles of western classically trained musicians. Many aspects of sound production (intonation, groove, spectral characteristics of instruments/voices, etc) is lost via this process.

Process of derivation and sound transformation is not directly apparent to the audience. The act of appropriation is not transparent.


\subsection{Spectral Information} 

\subsubsection{To generate sc}

\subsection{Computer Code}

Max patches, Computer Code.



\subsection{Real Performances} 


\subsection{Real-Time Plunderphonics}

Appropriation of audio signals from live music performances as material for a new composition

Creates a cognitive dissonance between audio and visuals.

The amount of processing of the audio signals is visible. The more processed the performances are, the more contrasting they will look in relationship with what is heard through speakers.

In contrast to acousmatic tradition, Real-Time Plunderphonics makes the process of appropriation transparent to the audience through the cognitive association between audio and visuals.

Changes relationship with the appropriated Other: The performer becomes an accomplice in the process of appropriation (or themselves). 

Deals with the problematic of the lack of visual clues and theatrical elements in electronic music performance by introducing a dynamic group of live performers and an interesting and unusual visual scenario.  

\subsubsection{Some ideas of how to plunder}

Get to know what and who you are plundering and figure why your are doing so before you decide how to plunder.(Know your performers, their music and why you want to work with them)

Appropriate and plunder yourself. 

Plundering not as central purpose of the creative process, but rather a tool for creating new idiosyncratic audio/visual result. 

Use ``from raw to cooked'' (L\'{e}vi-Strauss) techniques to create a narrative that navigates, in literary terms, between the �real� (actual performance) and the `surreal' (extreme processed audio).

Combinations of Real-Time Plunderphonics, (Real-Time) Musica Derivata and Sound Transformations

Use plunderphones as data: reprogram, not just remix.

Micro and macro plundering.

Use also Non Real-Time tools (Scores, Samples, etc.) if suitable. 

Using plunderphones as data

An example: Use FFT data of your plunderphone to trigger samples of recorded instruments.

\subsubsection{Micro and Macro Plundering}

Microplunderphonics

Plundering just microelements of sound. Not the whole spectrum of the original sound file. 

Generate noise with your plunderphones and use it instead of white noise for sound synthesis


Macroplundering

Appropriate a composition�s form. Use the structure as blueprint for a new composition. 

Use variables of the appropriated piece (pitch, dynamics, etc.) as control structures for new output.



\subsection{Musical postmodernism in the digital age}

resurgence of image / music quotations/references - first as reaction to the anti-mimetic
later with digital technology, easy reproduction, etc, etc => the use of images becomes the same as before the establishment aestetic regime : commodification, capitalism, DJ culture, digital quotations (in hip-hop, sound libaries, etc, etc)

I propose an attitude towards music appropriation similar to that of hacker communities and the open source initiative. Not with the purpose of suggesting a communist utopia, but of being consequent with my creative process. By giving away my music, recorded sounds and experiments, code, etc, through the net, I will hopefully instigate others to do so as well. If this attitude is followed, it could promote the organization of music cyber communities that would plunder, engage with and promote each other, hopefully producing more subversive types of music.

We are far from the Bourriaud�s utopia. The only people how have access to (artistic) shareware are commoditized people, mostly in western countries. Isn�t the DJ approach towards plunderphonics one that appropriates to make more profit and diminish costs only to thereafter feed back their product into the music industry system?

 
\subsection{Stratergies based on reshaping relationships in music making}

\subsubsection{Reshaping relationships in music making through technology?}

The introduction of electroacoustic resources into live musical performance has changed the relationship between the composer and the performer. 

The use of computer technology has also fostered new collaborative possibilities between performers of different cultures.

Musicians of different backgrounds (improvisation and notated music) and traditions (Western and non-Western) may now share the stage simultaneously and productively through technology; in spite of previously incompatible performance conventions.

Real-Time computer processing allows the possibility of using the audio signal (as well as other information - like MIDI) from several live performances simultaneously as building blocks for a composition.

\subsubsection{Crossing Cultural Borders?}

A discussion of Simon Emmerson's Crossing Cultural Boundaries through Technology. 
\v{Z}i\v{z}ek's view of Multiculturalism. 

Mention strategies: Algorithmic Score, Headphones, etc.

\subsection{Musica Derivata and Plunderphonics}

``A good composer does not imitate; he steals''       I. Stravinsky

Musica Derivata:

``music that is compositionally based on other music'' (K. Barlow) 


\subsection{plunderphomes, ideology and the use of references}

\begin{quote}
While some start up a prolonged lamentation for the lost image, others reopen their albums to rediscover the pure enchantment of images- that is, the alterity of the \emph{was}, between the pleasure of pure presence and the bit of the absolute Other.
\end{quote}
\begin{quote}
Evidence of exhibitions devoter to `images', but also the dialectic that affects each type of image and mixes its legitimations and powers with those of the other tow.
\end{quote}
Plunderphones reflect ideology . . . \v{Z}i\v{z}ek/Adorno but. . . . The artist can present their own view of these references by rearranging them modifying them. The plunderphonics artist doesn't necessarily adheres to the ideology of the appropriated material, but reflects it by the use of the plunderphones - how are they presented, modified, etc?  

\subsection{On Musical Appropriation}

What? 

Code, compositional tecniques, what piece of music? 
Do we plunder from the ``flea market or (the) airport shopping mall''? (N. Bourriaud). From the top 20 list - J. Oswald approach-, or from the hidden CDs at the back of the music store?

Who?

Music Industry? Pop/commercial? Historical (dead composers)? Music from different cultures? 

Appropriation of the Other. What relationship do we want to establish with the Other? Impersonal like the 1st/3rd World relationships?

Liberal multiculturalists approach? ``Other deprived of its Otherness (the idealized Other who dances fascinating dances and has an ecologically sound holistic approach to reality, while features like wife beating remain out of sight�)?'' (Slavoj \v{Z}i\v{z}ek, 2003)

Why?

For the meaning of the cultural object you are appropriating? For it�s symbolism? To suggest a metaphor?

For it�s use? ``Don�t look for the meaning, look for the use'' - L. Wittgenstein - for example for the sonic qualities of the appropriation (intonation, groove, etc.)

How? �

\subsection{Interpassivity}

\begin{quote}
Interpassivity, like interactivity, thus subverts the standard opposition between activity and passivity: if in interactivity (or the cunning of Reason), I am passive while being active through another, in interpassivity, I am active while being passive through another. More precisely, the term interactivity is currently used in two senses: (1) interacting with the medium, that is, not being just a passive consumer: (2) acting through another agent, so that my job is done, while I sit back and remain passive, just observing the game. While the opposite of the first mode of interactivity is also a kind of interpassivity, the mutual passivity of two subjects, like two lovers passively observing each other and merely enjoying each others presence, the proper notion of interpassivity aims at the reversal of the second meaning of interactivity: the distinguishing feature of interpassivity is that, in it, the subject is incessantly (frenetically even) active, while displacing on to another the fundamental passivity of his or her being.\footnote{From The Fantasy in Cyberspace by Slavoj \v{Z}i\v{z}ek}
\end{quote}

\label{ch:strategies}