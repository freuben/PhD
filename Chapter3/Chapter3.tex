\hypertarget{chapter3}{}
\chapter{Motivation}

In this chapter, I will attempt to examine...

\section{Redefining the Musical Subject?}

It appears that the dominant position at this moment regarding music is characterized by a skeptical and often cynical attitude towards new forms of thought in music. However, this attitude is dominant not without a reason: it has to do with the notion that today music is---as Alain Badiou has stated---`negatively defined.' Badiou clearly expresses this view in his essay entitled `Scholium: A Musical Variant of the Metaphysics of the Subject.'
\begin{quote}
Today, the music-world is negatively defined. The classical subject and its romantic avatars are entirely saturated, and it is not the plurality of `musics'---folklore, classicism, pop, exoticism, jazz and baroque reaction in the same festive bag---which will be able to resuscitate them. But the serial subject is equally unpromising, and has been for at least twenty years. Today's  musician, delivered over to the solitude of the interval---where the old coherent world of tonality together with the hard dodecaphonic world that produced its truth are scattered into unorganized bodies and vain ceremonies---can only heroically repeat, in his very works: `I go on, in order to think and push to their paradoxical radiance the reasons that I would have for not going on.'\footnote{Badiou,  `Scholium: A Musical Variant of the Metaphysics of the Subject', p. 89.} 
\end{quote}
Here, Badiou precisely delineates the situation in which so called `art' music or contemporary music is created and received today, where the only two main options seem to embrace either the joyful and permissive attitude towards mixing genres and styles now commonly ascribed to \emph{postmodernism} or the desolate notion of \emph{modernist} aesthetics that to this day heroically stands in `life support' for more than thirty years. These two positions also seem unable at present time to inspire a profound change in the way we create, perform, perceive and think about music; nor to respond to the original premise of the \emph{modernist} vision of the musical avant-garde, which establishes a connection between new forms of musical and political subjectivity. 

Ranci\`{e}re's analysis gives us strong theoretical tools that can help imagining new ways of reinvigorating the \emph{modernist} idea of the avant-garde in music without falling back to the misunderstandings that led to the `crisis of modernity.' Nevertheless, Ranci\`{e}re's notion of the \emph{avant-garde} is considerably different from the conventional one, and in order to understand his definition and relate it to music, it is important to separate it from its former association to a particular movement in music history. Even though the idea of the avant-garde in music emerged as it became associated to a group of `modernist' composers, the concept remains useful to us now only as a way of understanding the importance of the \emph{aesthetic regime} in the relationship between music and other types of subjectivity and forms of thought. To avoid further misunderstandings, one also needs to take special care and remember the clear differentiation Ranci\`{e}re makes between the \emph{strategic} and \emph{aesthetic} types of avant-garde.

\subsection{The \emph{Strategic} and \emph{Aesthetic} Types of Avant-garde in Music}

The \emph{strategic} type of avant-garde as manifested in music is one that can be associated to a particular group of people (composers, performers, critics and other people who make, think and/or listen to music), musical institution or movement that consolidates a type of subjectivity. It is important to remember that a common ideological position is what triggers the conception of this type of group.\footnote{Slovoj Zizek has repeatedly emphasized how ideology is not an abstract notion or theory one simply ascribes to, but a type of subjectivity that is reflected in the way we act, on how we behave and carry ourselves on a day-to-day basis. Therefore, a musical `movement' doesn't necessarily have to be one in which there is a `conscious' or openly declared agenda that follows a particular position of objectified consensus.} On the other hand,  the \emph{aesthetic} type of avant-garde as manifested in music is that which---through new ways of thinking and making music as expressed by the creation of new musical forms and structures---has the capacity to inspire and encourage new forms of thought about the life to come.  Furthermore, it is crucial that the \emph{strategic} type of avant-garde is not confused with the \emph{aesthetic} type in as much as it will lead to further misunderstandings within the music-world. 

It is important to note that one can find these two types of avant-gardes both in the musical and political spheres (as well as in the other artistic disciplines). Additionally, as they manifest themselves in music, the \emph{aesthetic} and \emph{strategic} types of avant-garde are intrinsically related; but only in as much as music is concerned.  This relationship becomes evident in the causality that exists between musical groups, institutions and movements; and the creation and reception of music. The \emph{strategic} avant-garde as manifested in music is therefore useful to the political sphere only as much as it contributes to the \emph{aesthetic} avant-garde---specifically as it provides a platform for the creation of `new sensible forms and structures.' Hence, the way in which the two types of avant-gardes dwell within music can not be directly compared to the way in which they reside in politics. Here lies another vital point in Ranci\`{e}re's enquiry: the \emph{strategic} type of avant-garde manifests itself \emph{differently} in music as it does in politics. Therefore, the activism of a musician or group of musicians as they become directly involved in politics does not reflect a relationship between music and politics, but only the involvement of a group of people---which happen to have the same occupation---in a political movement. The true relationship between music and politics is rather reflected in the \emph{aesthetic} type of avant-garde. This argument makes evident why it is misleading to attempt to identify a movement with concerns that are specific to music with a particular political affiliation or party. The position put forward by some critics of \emph{modernism} in music---which concludes that the emancipatory project which seeks the autonomy of music leads to totalitarianism---is therefore flawed.  

Moreover, I will claim that it is very important to consider the intrinsic relationship between the two types of avant-gardes, exclusively as they manifest themselves within music. The basis of this way of thinking stems from the assertion that the \emph{strategic} type of avant-garde has a considerable effect on the \emph{aesthetic} type in numerous significant ways. The impact that musical movements, institutions, ensembles and other organized groups of musicians and people dealing with music, have on the actual musical results, is often underrated. Too often, people involved in creating (particularly composers in my experience) and experiencing music avoid or forget how these strategic forms of collectivity condition and influence the aesthetic result. I will even go as far as to suggest that, in music, the type of subjectivity that is synthesized in the \emph{strategic} avant-garde is reflected or `embodied' in the \emph{aesthetic} avant-garde. That is to say, the ideology of the people involved in the creation, presentation and dissemination of music is expressed in the musical modes of action, production, perception and thought. Furthermore, the notion that the composer is the only person whose ideology is reflected in the music and that the \emph{musical work}\footnote{See Lydia Goehr, \emph{The Imaginary Museum of Musical Works: An Essay in the Philosophy of Music}, Oxford: Oxford University Press, 2007, for a thorough discussion on the philosophy of musical works.} is the only carrier of meaning---an idea that up to this moment is still widespread in western culture---is also misleading. In contrast to the more limited concept of a \emph{musical work}, I will therefore introduce to the notion of a \emph{musical result} as that which describes the complex set of percepts given by all aspects of a musical experience. These include for example: all sorts of aural and visual elements in music performance; the space and time in which music is performed; the way in which music is presented to the audience (including their role and participation in the musical experience); different modes of action in performance (performance practice) and composition (act of composing); the relationships established between composer, performer and audience; the context (cultural, sociological, political) in which music is presented; the way music is created, consumed and distributed; \emph{etc}. A particular kind of musical result consequently discloses a type of collective subjectivity which encompasses the ideology of the people involved in the music.\footnote{I am not implying however that the ideology of \emph{all} the people is represented \emph{equally} in the musical result. The question of how much an individual is represented widely depends on the role they take within the musical result and the audience's interpretation of it.} Additionally, within the musical result lies a system of elaborate symbols that synthesizes the relationships between the people involved in the collective act of music-making.

\subsubsection{\emph{Musicking}}

According to Christopher Small, these set of complex relationships that are formed between people involved in music is that which gives meaning to music. His interest lies particularly on the collective action surrounding music and defines this activity as \emph{musicking}. 
\begin{quote}
The act of \emph{musicking} establishes in the place where it is happening a set of relationships, and it is in those relationships that the meaning of the act lies. They are to be found not only between those organized sounds which are conventionally thought of as being the stuff of musical meaning but also between the people who are taking part . . . relashionships between person and person, between individual and society, between humanity and the natural world.\footnote{Christopher Small, \emph{Musicking: The Meanings of Performing and Listening}, Middletown, Connecticut: Wesleyan University Press, 1998, p. 13.}
\end{quote}
By giving priority to the verb \emph{to music}, as opposed to the noun \emph{music}, he also questions the notion of the \emph{musical work} and gives emphasis to the human action of \emph{musicking}. Small argues that music is not an object and that \emph{musical works} only give material for the musicians to perform, in contrast to the notion (developed as a consequence of western concert music) of performance only as a presentation of a \emph{musical work}. He also defines the verb \emph{to music} to include any type of action that contributes to a musical performance, which includes performing, listening, practicing, composing and dancing. He goes as far as to include actions such as selling and collecting tickets and cleaning the concert hall after a performance within his notion of \emph{musicking}. Therefore, \emph{musicking} encompasses all social relationships and actions that are related to music-making. Furthermore, he argues that \emph{musicking}, together with speaking, are characteristics that are at the very core of what makes us human.
 \begin{quote}
I am  certain, first, that to take part in a music act is of central importance to our very humanness, as important as taking part in the act of speech, which it so resembles (but from which it also differs in important ways), and second, that everyone, every normally endowed human being, is born with the gift of music no less than with the gift of speech.\footnote{Ibid., p. 8.}
\end{quote} 
Recent scientific studies in a variety of specialities including neuroscience, psychology, archaeology, anthropology and cognitive musicology have also pointed towards the same hypothesis. The idea put forward by Steven Pinker that music is `auditory cheesecake'---that it is only a byproduct of evolution and has no biological value for humans\footnote{See Steven Pinker, How the Mind Works, details missing...}---has been challenged recently within the scientific community.  These studies have shown how music plays an important role, amongst other things, in human communication, social bonding, cooperation, sexual selection, conveying emotions, phycological \mbox{well-being}, development of coordination and motor skills, expression of empathy, communication between infants and parents and exercising intelligence.\footnote{See Steven Mithen, \emph{The Singing Neanderthals: The Origins of Music, Language, Mind and Body}, London: Phoenix, 2006, for an overview of these studies.} In addition, various theories have emerged regarding the relationship between music and language; some of them even suggesting that \mbox{`proto-language'} (the predecessor of language) was a pre-linguistic, non-verbal form of communication that was a `musical' form of action and thought.\footnote{Ibid., pp. 147-150.} It appears that language and music have a similar evolutionary \mbox{starting-point} and the common purpose of communicating emotion and meaning through sound. Therefore, Small is right in suggesting that \emph{musicking}, like speaking, is at the core of being human and performs important social, cultural and biological functions. 

\hypertarget{musdef}{}
\subsection{The Definition of Music and the \emph{Ethical Regime}.}

The important functions music performs in the development of individuals and the way in which they establish and nurture relationships within a community is what defines music as a vital human act. Perhaps this is the reason why in the musical domain---going back to Ranci\`{e}re's notion of `the regimes of art'\footnote{See \hyperlink{artregimes}{pp. 4-8}.}---music is still defined as such within the \emph{ethical regime}. In other words, if one goes back to the question of why within music there is no change of identification with the break between the \emph{ethical} and \emph{poetic} regimes; I will suggest that it is because there is a strong ethical core implicit in the very meaning of \emph{what music is}. That is to say, as opposed to the definition of the other arts, the definition of music has been tied to the ethical functions that it performs for individuals and their communities. It is worth mentioning that only dance, like music, can also be defined as such within the \emph{ethical regime}, which points towards the deep-rooted relationship between both disciplines. On the contrary, other artistic disciplines including `fine' art, poetry and theater are identified as such only with the break between the \emph{ethical} and \emph{poetic} regimes.   

The ability that human beings have to communicate and perceive emotion and meaning through \emph{musicking} is also tied to music's identification and to the ethical functions it performs. It is by no coincidence that already in Ancient Greece, Aristotle observed that music has an immense power to change people's state of character and that different types of music affect audiences in different ways. According to Aristotle, music represents various types of emotions and actions that closely resemble those that the listener undergoes in reality as a result of the performance.\footnote{See Aristotle, `The Aims and Methods of Education in Music' in \emph{Politics}, Trans. Ernest Barker, Oxford: Oxford University Press, 1995, pp. 309-310.} It is as a consequence of this link between music and human experience, emotion and action that communities have attempted to regulate and evaluate music according to the ethical functions it performs. One could consequently argue that music that lies within the \emph{ethical regime} is evaluated for its ability to affect people in a way that is considered appropriate by the community, given a particular situation. This argument also points towards one of the reasons why labeling music as different `styles' or `genres' seems to be a dominant practice within communities: by knowing what kind of music to expect from a specific `style', it is possible to anticipate the type of experience the audience will go through. This is also one of the reasons why innovation in music has been discouraged and even censured by communities for centuries. The modification of musical styles within the perspective of the \emph{ethical regime} implies an unexpected change in ones experience and a potential threat to the community's consensus of what is considered to be the appropriate way in which people are to be affected by the music. Furthermore, innovation in music has been perceived as a political threat in the past since new forms of music produce new experiences that might stimulate behavior outside the political order. 

Plato, in his \emph{Republic} already warns about the danger that innovation in music might pose to the order of the State:
\begin{quote}
Put briefly, then, those charged with care of the city must hold fast to this, so that the city may not be corrupted unawares; but beyond all else, they must guard against innovation in gymnastic and music contrary to the established order, and to the best of their ability be on guard lest when someone says that people care more ``for the newest song on the singer's lips,'' the poet may be understand to mean not new songs but a new style of singing, and to comment it. One must not praise such a thing, nor so interpret the poet, but guard against changing to a new form of music, as endangering the whole. For styles of music are nowhere disturbed without disturbing the most important laws and customs of political order---as Damos says and I believe.\footnote{Plato, `Music and the Constitution' in \emph{The Republic}, Trans. R.E. Allen, New Haven: Yale University Press, 2006, p. 117}
\end{quote}
Therefore, the Platonic view regarding innovation in music is that it is threatening to the social agreements and political organization of the State. Even though the idea that innovation in music might endanger the political and social contracts of the community today might seem a bit far fetched, it still gives us a clue towards an attitude that up to this day is still widespread, that is: that innovation in music regarding its own rules, hierarchies, subject matter and genres is still received with reservation, suspicion and even fear amongst the wider community (if compared to the visual arts for example). In my opinion, this is due in the most part for to two main reasons. First, considering the implication that music performs certain ethical functions, innovation can be seen with skepticism as it could lead to confusion, uncertainty and even irritation, if the music ceases to perform the functions expected by the community successfully or does so less efficiently. Secondly, given the immersive and participatory (either by listening or performing) aspects implied in the definition of music that establishes a link between music and human action and experience, innovation in music can be associated with new and unpredictable experiences and behavior. Therefore, it is not surprising that some people would be distrustful in allowing themselves experience something they are not familiar with or are uncertain about.\footnote{On a related note: according to recent studies, most people stop acquiring new musical tastes by the time they are around twenty years old. This might be as a result that as people grow older, they seem less open to new experiences. See Daniel Levitin, `My Favorite Things' in \emph{This is Your Brain on Music: Understanding a Human Obsession}, London: Atlantic Books, 2006., pp. 231-233.}

\subsection{An Ethical Function within the \emph{Aesthetic Regime}?}

Going back to Ranci\`{e}re's notion of the regimes of art, if one considers the implicit ethical core in the definition of \emph{what music is} simultaneously with music that falls within the \emph{aesthetic regime}, one might run into a deadlock: if music is to be evaluated \emph{only} by the functions it already performs within the community (and innovation in music is only seen as a disruption from these functions), music that lays within the \emph{aesthetic regime} will not be understood or appreciated. To resolve this problem one needs to point towards the relationship that exists between music and other forms of thought and subjectivity. If music is evaluated and appreciated for its capacity to inspire new ideas, opinions, believes and desires, then one can argue that their is an ethical position implicit within the \emph{aesthetic regime}. In other words, their is an ethical function in itself in breaking with previous models of music making and in questioning the very notion of \emph{what music is}. This function is precisely that of imagining and experiencing through music, new forms of action, production, perception and thought.

Nevertheless, an agreement of trust needs to be established between the musical avant-garde and the community in order for the \emph{aesthetic regime} in music to be acknowledged and appreciated widely. Considering the ethical core implicit in music's definition, it is likely that the community will be unwilling to be open to new musical experiences if they fear that the ethical functions music already performs within the community will be disrupted or negatively altered. Therefore, this agreement needs to demonstrate that the purpose of creating new forms of music is not to betray its ethical functions, but to inspire and experience new forms of subjectivity---and this in itself is an important ethical function. Additionally, this agreement cannot only be reflected theoretically through verbal and written forms of public dissemination, but needs to be embedded within the musical result, if it is going to be understood by the wider community. Moreover, I will claim that the establishment of the \emph{aesthetic regime} in music and the redefinition of the `musician' as an occupation that questions the very notion of \emph{what music is}; has still not been spread out through the wider community. The reason, I believe, is that as a consequence of the practice of some musicians that can be associated with \emph{modernism} (mainly, those seeking music's `purity'  in composition through a militant anti-mimetic attitude focusing on abstract musical parameters and those who only advocate  and strive for `correctness' and `sterility' in performance practice) the agreement of trust between the wider community and the musical avant-garde has been broken. This is partly due to an attitude still influential in the musical avant-garde that does not address (or in some cases completely ignores) the most basic ethical functions that the community associates to music. Therefore, if the \emph{aesthetic regime} in music is to be acknowledged and appreciated widely, an agreement of trust needs to reestablished between the musical avant-garde and the community. 


\subsection{The Musical Avant-garde as Vehicle for Radical Change}

The acknowledgement of the \emph{aesthetic regime} in music within a wider community is of utmost relevance today. If one believes there is a connection between music and other forms of action, perception and thought, and simultaneously recognizes the need for radical change in other forms of human endeavor, one can acknowledge the potential of music in providing a space for inspiring and experiencing new forms of subjectivity for a life to come. Furthermore, if the wider community understands this link and at the same time maintains an agreement of trust with the musical avant-garde, this will lead to further reflection on how to bring change to other forms of human knowledge and action and inspire new alternatives to the current state-of-affairs. Additionally, given the connection that exists between the \emph{aesthetic} avant-garde in music and politics, one can assume that an ingenious and vibrant musical avant-garde can only contribute to a strong and active political avant-garde. Given the bleak prospects that the current political climate has to deliver radical reform that will tackle problems that are catastrophic in scope, the importance of finding new alternatives to the current political models is crucial. As we face colossal problems---like the ecological catastrophe, unsustainability, overpopulation, economic crises, world poverty and inequality---that put in danger our very survival as a spices, it is critical to find insightful solutions to these problems and to \mbox{imagine} and implement new political and social models for the future. I believe music can contribute to this change as it provides the possibility of immersion into new experiences. Taking in consideration the evolutionary connection between music, emotion an thought as well as Aristotle's observation of music's ability to affect human beings, 

Music also provides a model by which musical thought and subjectivity are implemented into musical activity and production that can serve as a metaphor for modes of action in the political sphere. 

In other words, the way in which a musical idea is put into practice

People Getting together and gathering around a performance! Social occasion. knowledge of a common idea/subjectivity - > encouragement, "Revolution" StevePinker (the thought of stuff: language as a window into human nature"
 
 as well as it can provide an aesthetic metaphor for the political sphere in the way mode of implementation of musical thought and subjectivity into action.
 


Music is Transformative
Music is a vehicle...


 experiences and actions that embody new forms of subjectivity.

I will suggest that a potential exists for the musical avant-garde if an agreement of trust can be established with the community. New types of music may inspire new forms of objective thought, but most importantly, it may also provide the potential for immersing oneself into new experiences and actions that embody new forms of subjectivity. A link between the political and musical avant-garde

Why?


Emancipatory potential of music therefore lies in the possibility of changing music's ethical functions

As a consequence of this agreement of trust established between the musical avant-garde and a wider community, I will suggest that music has the potential for providing inspiration to the political avant-garde not only through new forms and structures, but also through experience and action. 


I will also argue that music has a particular emancipatory potential given its particular position within the artistic regimes.\footnote{See \hyperlink{artregimes}{pp. 4-8} for a discussion on Ranci\`{e}re's ideas regarding the artistic regimes.} 

%- \emph{aesthetic regime} to performers?? confusion with aesthetic regime again... it is not necessarily that performers have a role in composition...or improvisation...they are still in the representative regime...they still use modes and resemble performance conventions (in improv) ... that does not solve the `problem' and they are not going into the aesthetic regime., for that to happen performers have to reinvent what they do, they should do whatever they consider they should do and what ever they think performers should do.



\subsection{Strategic Views on Aesthetic Forms}

If a positive redefinition of music is to take place, and an agreement of trust to be reestablished between the musical avant-garde and the wider community; it is crucial to examine the fundamental aspects of how music is created, performed, presented and disseminated today. This includes a significant revision and modification of the \emph{strategic} forms of collectivity in music. In other words, in order to reinvigorate (within the musical sphere) the \emph{aesthetic} type of avant-garde, the \emph{strategic} type of avant-garde also needs to be rethought and reworked. Furthermore, if the agreement of trust between the musical avant-garde and community is to be regained, I believe it is important for the \emph{strategic} avant-garde to consider the ethical core implicit in the definition of music in parallel with a strong desire towards innovation and change in all aspects of music-making. In other words, while acknowledging the audience and their perception of what the fundamental ethical functions of music are---by making them experience something that they would associate with their conception of music-making within the \emph{musical result}---at the same time challenging those very notions and putting into question the fundamental aspects of how music is created and received. Furthermore, if one subscribes to this position, one should consider the role musical groups, institutions, ensembles, industry and movements might have in the musical result he is involved with, in order to determine whether these groups might help in the establishment of new \emph{aesthetic} forms. Moreover, it is vital to consider the context, time, space and audience where the music is to be presented as this too has a direct causality with the \emph{aesthetic} result and its visibility, and plays a significant part in the disclosure of a particular type of experience. 

How to reestablish the agreement? Appropriation/Ideology 

\subsubsection{Innovation, innovation, innovation...}

How to innovate?
 
\subsection{Technology and Innovation}

\begin{quote}
In other words, if in the past---even the distant past---music was often the testing bench and the stimulus for scientific research, and thus music tended to draw scientific knowledge to it, in more recent years you get the impression that it's now science that draws music to it and takes possession of it. Indeed, you often get the impression that a scientific creativity applicable to music has substituted itself for musical creativity, and that musical thought has regressed to the level of the (invariably squalid) opinions that an electronic engineer from Bell Telephone or a Stanford ``software man'' may have about music. . . . Thus many of the more sensitive musicians quickly realized that it was as easy as it was superfluous to produce new sounds that were not the product of musical thought, just as it's easy nowadays to develop and `improve'' the technologies of electronics music when there are devoid of any real and profound \emph{raison d'\^{e}tre}.\footnote{Luciano Berio, \emph{Two Interviews with Rossana Dalmonte and B\'{a}lint Andr\'{a}s Varga},  Ed. and Trans. David Osmond-Smith, London: Marion Bowars, 1985, pp. 121,122.} 
\end{quote}

\begin{quote}
It was recognized, for example, that the spectacle of a public gathered together to listen to loudspeakers was not a particularly cheerful one, and that, yet again, the experience of public musical listening was made up of may different conventions, and was rooted in many different aspects of social and cultural life: it was not made up merely of a piece, a musical object to listen to, even if it proposed ``new sounds''. By its very nature, a piece of music by itself cannot easily transform listening conventions and socio-musical relations in general.\footnote{Ibid., pp. 122,123.}
\end{quote}

\section {Appropriation and Ideology in Music}

\begin{quote}
The contemporary era constantly proclaims itself as post-ideological, but this denial of ideology only provides the ultimate proof that we are more than ever embedded in ideology. Ideology is always a field of struggle---among other things, the struggle for appropriating past traditions.\footnote{Slavoj Zizek, ``It's Ideology, Stupid!'', in \emph{First As Tragedy, Then as Farce}, London: Verso, 2009, p. 37.}
\end{quote}
Start on appropriation and past traditions...


My approach to appropriation??
\begin{quote}
Characters on stage should be flat, like clothes in a fashion show: what you get should be no more than what you see. Psychological realism is repulsive, because it allows us to escape unpalatable reality by taking shelter in the ``luxuriousness'' of personality, losing ourselves in the depth of individual character. The writer's task is to block this manoeuvre, to chase us off to a point from which we can view the horror with a dispassionate eye.\footnote{Elfriede Jelinek, quoted in Slavoj Zizek, \emph{First As Tragedy, Then as Farce}, p. 40.}
\end{quote}

\begin{quote}
Consumption is simultaneously also production, just as in nature the production of a plant involves the consumption of elemental forces and chemical material\footnote{Karl Marx}
\end{quote}

\begin{quote}
Starting with the language imposed upon us (the system of production), we construct our own sentences (acts of everyday life), thereby reappropriating for ourselves, through these clandestine microbricolages, the last word in the productive chain.\footnote{Nicolas  Bourriaud, \emph{Postproduction. Culture as Screenplay: How Art Reprograms the World}, New York: Lukas and Sternberg, 2005.}
\end{quote}

\subsection{Appropriation and Postproduction in the Digital Age} 

\begin{quote}
By listening to music or reading a book, we produce new material, we become producers. And each day we benefit from more ways in which to organize this production: remote controls, VCRs, computers, MP3s, tools that allow us to select, reconstruct, and edit. Postproduction artists are agents of this evolution, the specialized workers of cultural reappropriation.\footnote{Ibid. p. ?}
\end{quote}

\begin{quote}
Throughout the eighties, the democratization of computers and the appearance of sampling allowed for the emergence of a new cultural configuration, whose figures are the programmer and DJ. The remixer has become more important than the instrumentalist, the rave more exciting than the concert hall. The supremacy of cultures of appropriation and the reprocessing of forms calls for an ethics: to paraphrase Philippe Thomas, artworks belong to everyone. Contemporary art tends to abolish the ownership of forms, or in any case to shake up the old jurisprudence. Are we heading toward a culture that would do away with copyright in favor of a policy allowing free access to works, a sort of blueprint for a communism of forms?\footnote{Ibid. p. ?}
\end{quote}

%\subsection{critisisms} 
\subsection{The liberal-comunists: Open Source, etc.} 

There is no music by John Oswald on the net free to download. Hypocrisy from the appropriator? Or does he fall into the logic of late-capitalism - �no communism of forms�? �I plunder but don�t plunder me. Or, at least not for free��? 

I propose an attitude towards music appropriation similar to that of hacker communities and the open source initiative. Not with the purpose of suggesting a communist utopia, but of being consequent with my creative process. By giving away my music, recorded sounds and experiments, code, etc, through the net, I will hopefully instigate others to do so as well. If this attitude is followed, it could promote the organization of music cyber communities that would plunder, engage with and promote each other, hopefully producing more subversive types of music.

We are far from the Bourriaud�s utopia. The only people how have access to (artistic) shareware are commoditized people, mostly in western countries. Isn�t the DJ approach towards plunderphonics one that appropriates to make more profit and diminish costs only to thereafter feed back their product into the music industry system?


\label{ch:motivation}