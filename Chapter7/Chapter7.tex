\hypertarget{chapter7}{}
\chapter{Compositions}

\section{E-tudes}

\href{http://www.ranchonotorious.org/freuben/e-tudes}{E-tudes} is a set of electronic \'{e}tudes for six stage pianos, live electronics and Disklavier\footnote{In case a Disklavier is not available, it is possible to use an electronic stage piano or a sampler with piano sounds.}. These compositions where written for the ensemble \href{http://www.pianocircus.com/}{\textbf{piano}\emph{circus}}\footnote{The original six piano ensemble was formed in 1989 to perform Steve Reich's Six Pianos. Since then the original members have changed and now comprise of David Appleton, Adam Caird, Kate Halsall, Semra Kurutaç, Paul Cassidy and Dawn Hardwic.} for a project that became a two-year collaboration and lead to two performances\footnote{Enterprise 08 Festival, The Space, London, 14 May, 2008, and The Sound Source, Kings Place, London, 9 July, 2009, sponsored by the PRS Foundation Live Connections scheme and Sound and Music.}. What initially attracted me to this ensemble was it's very particular instrumentation consisting of six electronic stage pianos. I thought this to be a suitable platform to experiment with the concept of \emph{Real-time Plunderphonics}\footnote{See pp. xx-xx for a discussion about this concept.} considering that these instruments are electronic and therefore produce no considerable audible acoustic sound.\footnote{The only acoustic sounds that can be heard are the keyclicks produced by the physical contact with the stage pianos while playing. This noise is slightly audible mostly when the sounds in the speakers are quiet or at moments where the speakers are silent.} Therefore, if the original music that the pianists play would be processed by the computer, the live-treatment of the sounds would be the only audible result---it would not be necessary to deal with the acoustic sound that would sound regardless of the computer processing if the instruments were acoustic. In other words, the music being played would be hidden from the audience when stripped away from its original sound. Another advantage that I found in using these keyboards is that I could not only use audio signals, but also the Midi messages as building blocks for the composition. Considering these opportunities for experimentation, I embarked in this large project which I plan to continue in the long-run. 

Like a book of \'{e}tudes from the repertoire, E-tudes consists of a set of pieces that can be performed together at the same event or individually as separate short pieces. At present time, I have completed four `e-tudes', and as an ongoing project I will continue adding new pieces to the group. The way in which these `e-tudes' are presented can also be modular: depending on the set of circumstances for a given event, they can be presented either as a concert performance or an installation with perforative elements. In the installation version, the audience walks into, out of, and around the area surrounding the musicians and has creative control over how they want to experience the performance. By choosing between listening to the speakers in the room or through headphones generating different outputs and distributed through the performance space, each member of the audience fabricates their own version of the piece. Therefore, in the installation version their are various possible outputs generated by the processing of the

Hello sentence.

	simultaneously a performance and an installation. A single multilayered composition will be performed at various times over the course of the event.

The ensemble of six stage pianos is placed in hexagonal formation and divided into two subgroups. The first subgroup consisting of three pianists are asked to select \'{e}tudes from the western piano repertoire \footnote{Examples of these are \'{e}tudes by Chopin, Ligeti and Debussy, to mention just a few.} and are to play them in the order of their choice during the duration of the performance. The second subgroup consisting of the remaining three pianists perform together from \emph{The Sixth Book of Madrigals} by Don Carlo Gesualdo da Venosa (1566-1613). 

The pianists playing the madrigals send Midi information to two laptops that will transform the audio signal from the etudes and schedule the digital signal processing events. The audience will not be able to hear in the room what the pianists are playing as the stage pianos do not produce an acoustic sound. The seventh performer -the composer himself- will speak the Madrigals' text through a microphone and the spectral information from this signal will be used to process the final audio output and to trigger other sound events. The composer will also play a Midi controller and will not have a fixed score, leaving space for an improvisational element within the human/computer interaction. Finally, through the analysis of all the inputs the computers will send Midi messages to a Disklavier (mechanical piano) that will play the role of "virtual soloist" for the performance. In the room one will be able to hear the final result of the creative process of combining the simultaneous performances in diverse arrangements. The headphones that will be spread through the performance space will portray the inner life of the performance sounding in the room and reveal the inner layers of computer processing as well as the appropriated compositions.

E-tudes also challenges the audience by questioning traditional performance practice and creating a cognitive dissonance: what you see is not necessarily what you hear, and certainly not what your past experience leads you to expect.

	Piano Circus, an ensemble featuring six pianists: Kate Halsall, David Appleton, Adam Caird, Semra Kurutac, Helen Reid and Graham Rix, will perform the piece. They will be playing on Roland RD700 Stage Pianos. The composer, Federico Reuben, will join them performing live-electronics: laptops, Midi-controllers, microphone and mixer. 

	The ensemble will stay in their usual hexagonal formation but will be divided into two subgroups. Three pianists will choose etudes that are established in the piano repertoire (Chopin, Ligeti, etc.) and perform them whenever they want during the duration of the piece. They will be monitored individually through headphones. The other three pianists will perform together from the 6th book of Madrigals by Don Carlo Gesualdo da Venosa (1566-1613) and will be able to hear each other through headphone monitoring. The pianists playing the Gesualdo will send Midi information to two laptops that will transform the audio signal from the etudes and schedule the digital signal processing events. The audience will not be able to hear in the room what the pianists are playing as the stage pianos do not produce an acoustic sound. The seventh performer -the composer himself- will speak the Madrigals' text through a microphone and the spectral information from this signal will be used to process the final audio output and to trigger other sound events. The composer will also play a Midi controller and will not have a fixed score, leaving space for an improvisational element within the human/computer interaction. Finally, through the analysis of all the inputs the computers will send Midi messages to a Disklavier (mechanical piano) that will play the role of "virtual soloist" for the performance. In the room one will be able to hear the final result of the creative process of combining the simultaneous performances in diverse arrangements. The headphones that will be spread through the performance space will portray the inner life of the performance sounding in the room and reveal the inner layers of computer processing as well as the appropriated compositions.

	The music will be specifically composed for this event and for Jerwood Space. Since the piece is conceived as an installation as well as a performance it is best suited to a space that encourages moving around and interacting with the work. In contrast to the concert hall, where the audience is locked to a single location, the space should promote interaction and invite the audience to pick up the headphones, which will be spread around. People should also be able to walk around and experience the piece from several locations and focus on various aspects of the different performances taking place. This venue offers all of these possibilities as well as giving the opportunity to go out and re-enter the space during the duration of the event. One can argue that these elements are fundamental for a piece that seeks to form a relationship with the listener and thus, it remains important that this event take place in this type of setting. 
 
	E-tudes questions the traditional role and relationships between performer, composer and listener and gives a unique and innovative approach to the use of “found objects”. The composer in this piece does not communicate with the performers by writing a score or by teaching them the music ‘by ear’ as in previous performance practice conventions. He even lets the performers decide which pieces to play within a given repertoire. Therefore, the creative role of the composer is not to provide the music the performer should play but rather, in Oswaldian terms, to plunder their audio signal. On the other hand, E-tudes differentiates itself from John Oswald’s ‘Plunderphonics’ in that the plundering occurs in a live situation and that makes the performer an accomplice in the process of appropriation (of themselves). In a way, since E-tudes appropriates several live performances simultaneously, it proposes the notion of plundering in real-time, or ‘Real-Time Plunderphonics’. It is therefore important that the event take place in a live situation, as the theatrical effect of being plundered will be evident visually in relationship to the audio. Consequently, the amount of processing of the audio signals will be visible to the audience and the more processed the performances are, the more contrasting they will look in relationship to what is heard through the speakers. In E-tudes, this premise is consciously used to create a narrative that navigates, in literary terms, between the ‘real’ (actual performance) and the ‘surreal’ (more extreme processed audio). In contrast with the acousmatic tradition (music presented through loudspeakers in a fixed medium where the sound sources are not visible), the live performance makes the process of appropriation transparent to its audience as a result of the cognitive association between audio and visuals. In an acousmatic approach, a sound that is radically processed loses its characteristics and therefore the cognitive relationship between source and result may be lost. On the other hand, if the source is exposed visually in a live performance, the audience will have more audio/visual links and one may suppose that the audio processing could be even more extreme without losing the association with the source. 
	Furthermore, E-tudes’ approach is atypical in relationship to ‘Plunderphonics’ or other music that borrows found material (for example, by musical quotation) in that plundering is not the central purpose of the creative process, but rather a tool for creating a new idiosyncratic audio/visual result. This difference is rather important since it addresses the question inherent in the ambivalence of plundering oneself to create something new as opposed to performing something new in an immediate and direct fashion. Therefore, the idiosyncratic result justifies the conscious participation of the performer in a piece in which what he or she plays is not directly heard by the audience.  This position proposes a new relationship between performer and composer and it also presents a new approach to composition. The composer’s role is not to establish direct communication with the performer (through a score or oral tradition) but rather to use live audio signals of existing music as building blocks to create a new work. All of this is achieved by writing computer software (using SuperCollider 3 – a programming language specialized in audio applications) specifically for the piece. Moreover, E-tudes takes a didactic attitude toward the process of appropriation by giving the listener access to the processed and unprocessed building blocks to show the different layers within the composition, not with the intention of being explicit, but to engage and establish a relationship with the listener. Finally, this composition combines the use of improvisation and generative music to have an unfixed output that changes for each performance of the work. This enables the piece to run in a loop during a long extended time frame without repeating itself. Every time the piece will be played not only will the audiences’ experiences differ, because of their own choices, but also the content of the piece itself will vary.
	E-tudes takes many elements used before in electronic music and live performance such as improvisation, appropriation, generative music, installation and traditional performance practice, and by combining them points to a development in performing with live electronics. By introducing a dynamic group of live performers and an appealing and interesting visual scenario, this event deals with the problematic of the lack of visual clues and theatrical elements that live electronics performance has faced since its beginning. Hopefully, it will also encourage other creators that deal with live electronics to think seriously about the visual, theatrical and ritualistic aspects of performance. This composition will also contribute to instigating awareness within the contemporary music community on how the presentation of a piece can be as crucial as the sound. It also proposes that the creator is able to innovate by searching for new ways that the audience relates to the work. The event will also contribute to the creative development of the artists because it will give them the opportunity to try out and experiment on the various interactive and performative aspects of the piece and later examine and evaluate how these processes may be improved.

Other important aspects about E-tudes:

Performance/Installation 

Audience will have creative control over how they want to experience the performance.

By choosing between listening to the speakers in the room or through headphones they will fabricate their own version of the piece.

Didactic attitude towards appropriation: listener will access processed and unprocessed building blocks, not with the intention of being explicit, but to engage and establish a relationship with the audience.

Relational aspect: it proposes the idea that one may innovate by searching for new ways that the audience relates to the work.

Elements of improvisation and generative music. Every time the piece will be played not only will the audience’s experience differ, because of their own choices, but also the content of the piece itself will vary.

\section{On Violence}

This composition attempts to explore the aesthetics of violence and reflect on different manifestations of violence. It is also inspired by slovenian philosopher Slavoj Zizek's ideas about violence. Zizek  categorizes violence into two main types: subjective and objective violence. Subjective violence is clearly identifiable by an agent, for example acts of terror or crime, and it is perceived as a clear interruption of the normal state of things. On the other hand, objective violence is  violence that is inherent in the social fabric and it is hard to see and experience for the advantaged classes or countries. What Zizek argues is that objective violence is inherent within the social "balance" and it is objective violence which triggers acts of subjective violence. Furthermore, Zizek identifies two types of objective violence: symbolic and systematic violence. Systematic violence is manifested through our economic and political systems that in order to give the idea of a normal smooth running of things, exert systematic violence on large groups of people. Symbolic violence is related to and included within systematic violence but it is specific to violence expressed through language and other symbolic systems (like music). Zizek goes further to argue that the forms of symbolic violence are actually based on and manifested by the symbolic systems as such. 

\section{Zizek?}

Zizek? is a computer-mediated improvisation that gives a live alternative soundtrack to the Zizek! (2005) movie. Each performer has a laptop in front of them. The laptops are connected through a network by which the composer guides the improvisers by sending them written directions, animations (moving graphical notation) and through headphones, an aural score that consists of sound and music derived from the audio of the film. 

Alexander Hawkins � piano. www.alexanderhawkins.com
Dominic Lash � bass. www.dominiclash.co.uk
Javier Carmona � drums. www.myspace.com/javiercarmonam
Federico Reuben � computer. www.myspace.com/freupinta

\section{FreuPinta}

%\subsection{Simulation Series}
%\subsection{Occupation Series}
%\subsection{Transgression Series}

\section{Improvisations}

%\subsection{Horatio Oratorio}
%\subsection{Perro-chimp}
%\subsection{Too Hot to Handel}
%\subsection{Mowgli}

\label{ch:compositions}