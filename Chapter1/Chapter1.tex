\chapter{Preface}

\pagenumbering{arabic}

Writing about music can be a complex endeavor. Theodor Adorno . . .

Writing about one's own music might become an even riskier venture.

Little intro... 

Critical Theory - approach, etc.

This commentary takes as a starting point that music can not simply be defined as the resulting sounds of a musical performance. Music is much more than that... it includes the relationships, social act, etc... "Music" as a definition that can change (it already has historically)...

Multidisciplinary approach...

On methodology... Meta-commentary, etc.

In \hyperlink{chapter2}{Chapter 2}, I will give theoretical background based on Ranci\`{e}re's views on the relationship between aesthetics and politics. Additionally, I undertake the task of trying to apply his concepts as they pertain to music. 

In \hyperlink{chapter3}{Chapter 3}---motivation

In \hyperlink{chapter4}{Chapter 4}---technology and musical strategies 

In \hyperlink{chapter5}{Chapter 5}---appropriation

In \hyperlink{chapter6}{Chapter 6}---computer applications

In \hyperlink{chapter7}{Chapter 7}---compositions

I would like to thank:

Parents 

Eszter Takacs

Family: specially bro

Supervisors: Richard Barrett and Christopher Fox

Aleksander Kolkowski

Nicholas Peters

Computer: SuperCollider community, David Plans Casal, Gabriel Montagn\'e

Impro: Javier Carmona, Dominic Lash, Alexander Hawkins

Performers: Sarah Nicols, Piano Circus for their time and effort devoted

Commentary: Philip Somervell, Adam de la Cour

Conversations: Steve Potter, Ed McKeon, Mauricio Pauly, Assaf Gidron

The Hague Colleagues: Jasna Velickovic, Ronald Boersen, Dganit Elyakim, Ophir Ilzetzki

\label{ch:preface}
