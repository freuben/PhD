\chapter{Introduction}

In this chapter, I will attempt to tackle different questions concerning the aesthetics surrounding my work with the purpose of introducing a conceptual framework that hopefully will situate it within a wider cultural and philosophical context. Elaborate a bit more...

%I will start by addressing some concerns regarding the notion of modernity within the musical discourse of western avant-garde composers and its approach towards tonality and making references to other musics. I will argue, through the work of french philosopher Jaques Ranci\`{e}re, that the departure from making references to other musics either from western classical tradition or other ethnic backgrounds, was not part of the initial modernist project, but a misunderstanding of its original premises founded on the political desire of establishing what he calls the \emph{aesthetic regime} of the arts. Nevertheless, before dwelling into this debate, in order to fully understand this confusion and later grasp certain concepts that will be useful in understanding some of my artistic motivations, I will examine Ranci\`{e}re's idiosyncratic and revealing view on aesthetics and its relationship to politics.
%Later, I will continue by discussing the position of the first generation of composers of the so-called \emph{postmodernist} movement and their motivation to reintroduce references of other musics in their own work as a reaction to the anti-representative camp. Thereafter, I will continue by discussing certain aspects about music appropriation and the role of technology.

\section{Ranciere and the reevaluation of the \mbox{notion} of modernity}

Jaques Ranci\`{e}re in his book \emph{The Politics of Aesthetics} examines the relationship between the concept of modernity and the break from figurative representation in the visual arts. He argues that the departure from representation of images through figurative means is often confused with aesthetic modernity, which is specific to a single regime of the arts. That is, ``a specific type of connection between ways of producing works of art or developing practices, forms of visibility that disclose them, and ways of conceptualizing the former and the later.''\footnote{Jaques Ranci\`{e}re, `The Distribution of the Sensible', in \emph{The Politics of Aesthetics}, Trans. Gabriel Rockhill, London: Continuum, 2004, p. 20.} If on is to think about this confusion that is associated with the concept of \emph{modernism} in the realm of music, some questions come into mind: Does this confusion apply to the musical domain and if so how does it manifest itself? Is it possible to talk about representation in music and if so within what context? Could one compare the breaking from figurative representation to the departure from tonality at the beginning of the twentieth century? Has `the musician' gone through a corresponding redefinition of \emph{what is expected} from him by the community the same way as `the fine artist' has through the process of modernisation?

In the following discussion, I will attempt to read Ranci\`{e}re's text as applied to music not only with the purpose of tracing parallels and discrepancies between music and the other arts, but to try to find out something particular about music itself. Also, I will venture to examine the limitations of the notion of modernity within music and its relationship to the wider modernist political project.

\subsection{The distribution of the sensible}

Before starting our discussion on the notion of modernity and its political and aesthetic consequences, first I will try to examine the relationship of aesthetics and politics in the work of Ranci\`{e}re. According to Ranci\`{e}re, the political and the aesthetic spheres are intrinsically linked through what he calls `The distribution of the sensible'. 
\begin{quote}
I call the distribution of the sensible the system of self-evident facts of sense perception that simultaneously discloses the existence of something in common and the delimitations that define the respective parts and positions within it. A distribution of the sensible therefore establishes at one and the same time something common that is shared and exclusive parts. This apportionment of parts and positions is based on a distribution of spaces, times, and forms of activity that determines the very manner in which something in common lends itself to participation and in what way various individuals have a part in this distribution.\footnote{Ibid., p. 12.}
\end{quote}
It is precisely this system of division of spaces, times and forms of activity that defines aesthetics and is also at the heart of politics. Therefore, aesthetics takes part in the political act of governing and in determining who the rulers are and how they come to power; as well as how the commons are distributed within a community. Here though, Ranci\`{e}re points out, that in order to make the relationship between politics and aesthetics, one must understand aesthetics ``in a Kantian sense---re-examined perhaps by Foucault---as the system of \emph{a priori} forms determining what presents itself to sense experience.''\footnote{Ibid., p. 13.} Aesthetics therefore should be seen here beyond the conventional view as strictly belonging to the confines of art and should not be seen merely as the `aesthetic practices' manifested in different artistic disciplines.  In contrast, in order to think of aesthetics in a context that could be applied outside of the arts, it requires its abstraction as modes of action, production, perception and thought; a system of ``delimitation of spaces and times, of the visible and the invisible, of speech and noise, that simultaneously determines the place and the stakes of politics as a form of experience.''\footnote{Ibid.}
Therefore, through the work of Ranci\`{e}re, it is possible to think of aesthetics in politics with a broader understanding of aesthetics as the distribution of the sensible. Moreover, for Ranci\`{e}re, `aesthetic practices' that disclose visibility in artistic practices reveal `ways of doing and making' that exist and have visibility within the community. There are different manifestations of this practices that confine an aesthetic distribution.

\begin{quote}
This forms define the way in which works of art or performances are `involved in politics', whatever may otherwise be the guiding intentions, artists' social modes of integration, or the manner in which artistic forms reflect social structures or movements. . . . In this way, a sensible politicity exists that is immediately attributed to the major forms of aesthetic distribution such as theater, the page, or the chorus. There `politics' obey their own proper logic, and they offer their services in very different contexts and time periods.\footnote{Ibid., pp. 14-15.} 
\end{quote}
Consequently, it could be argued that there is an inherent political core in the way this artistic forms are constituted and within them lays a political project that renders the distribution of what is known to the community, the way it is organized and what remains visible or invisible.    

\subsection{The regimes of art}

In order to understand Ranci\`{e}re's reevaluation of the notion of modernity one must first understand what he calls the three `regimes of art', which are modes of identification and articulation between `ways of doing and making' and forms of visibility, as well as their conceptualization. In other words, the `regimes of art' simply distinguish different ways of making and thinking about `art' and how it is perceived.

\subsubsection{The ethical regime of images and the poetic regime of art}

To begin with, Ranci\`{e}re defines the \emph{ethical regime of images} as the pragmatic Platonic\footnote{As in Plato's \emph{The Republic}---not sure about this one. . .} notion of the use and distribution of images in relationship to the community's \emph{ethos}. This regime therefore uses images as `true' imitations of the original and are distributed and valued by their purpose of educating the community in accordance to it's social order. Therefore, within this regime `art' is not evaluated by qualities within itself but by their purpose in the community. He goes on to define a \emph{poetic regime of art} (also referred to as \emph{representative regime of art}) as that which breaks away from the \emph{ethical regime of images} and values the arts in terms of their own \emph{substance}.

\begin{quote}
I call this regime \emph{poetic} in the sense that it identifies the arts---what the Classical Age would later call the `fine arts'---within a classification of `ways of doing and making', and it consequently defines proper `ways of doing and making' as well as means of assessing imitations. I call it \emph{representative} insofar as it is the notion of representation or \emph{mim\={e}sis} that organizes these ways of doing, making, seeing and judging. Once again, however, \emph{mim\={e}sis} is not the law that brings the arts under the yoke of resemblance. It is first of all a fold in the distribution of `ways of doing and making' as well as in social occupations, a fold that renders the arts visible. It is not an artistic process but a regime of visibility regarding the arts.\footnote{Ibid., p. 22.}
\end{quote}

Here, one comes to the first deadlock in reading Ranci\`{e}re's notion of the `regimes of art' as applied to music. First, in order to understand the difference between the \emph{ethical regime of images} and the \emph{representative regime of art} outside the domain of the visual and fine arts one must remember that music not only has different social functions and visibility, but within it's unique organization it has particular `ways of doing and making' that are specific to it's own disipline. Even though music occupies a different and particular position in the ways of distributing the sensible, I will continue to argue that it is still possible to refer to the \emph{ethical} and the \emph{poetic} regimes in music. 

Following Ranci\`{e}re categorization, I will refer to music within the \emph{ethical regime} as music that is made, heard and judged for it's purpose within the community. By this, I mean music that is not assessed by it own qualities---or as Ranci\`{e}re would say `by it's own \emph{substance}'---but by the purpose it performs within the community. Examples of this in western tradition would include church, court and military music, to mention just a few. It is easy to find music that falls within the \emph{ethical regime} in other cultures where in some cases music is not even differentiated from other disciplines, like dance or storytelling, and is performed (in some cultures everyone partakes in music-making) and valued by members of the group by it's communal and ceremonial purposes (celebration, mourning, war, etc). Of course, one can still find many examples of the \emph{ethical regime} today in music for theater, dance, television, films and religious purposes. I am also of the opinion that a large quantity of \emph{pop music} is also not appreciated by it's own `musical' attributes but is judged by it's purpose of making profit within a consumer society (selling records, tickets and other merchandize). That said, I want to make clear that I am not attempting to devalorize or make a value judgment about music that falls within the \emph{ethical regime}. Furthermore, some music might also qualify within more than one regime simultaneously.

I will define music that falls within the \emph{poetic regime} as that which is appreciated for its own \emph{substance} but still follows or imitates a model.\footnote{By model I not only mean the written but also the unwritten rules in music performance and composition. The written rules could be for example tretises of harmony and orchestration whereas the unwritten rules could be performance practices and conventions in composition and improvisation, too name a few.} Namely, music that is judged by it's own `musical' qualities, and that is made with the main purpose of been listened to and evaluated by it's own subject matter. This music would be \emph{representative} insofar as it imitates or resembles a musical model (for example rules of harmony, counterpoint or sonata form, to mention just a few). A lot of western `concert music' would follow in this category in that it is made, heard and valued for it's `musical' qualities and judged as good or bad, adequate or inadequate, satisfactory or not, based on how the performer or composer follows certain models---in the case of the performer, models of performance practice, and in the case of the composer, compositional models such as chord progressions, \mbox{voice-leading}, musical themes, variations, etc. 

It is interesting to note that even though in the visual arts, the breaking from the \emph{ethical regime of images} and the establishment of the \emph{poetic regime of art} is what separates the `fine arts' from other modes and techniques of production (of images, shapes, objects, etc), in music there is not such a change in definition. That is to say, in the visual arts this break between \emph{ethical} and \emph{poetic} regimes identifies the arts as such but in music it does not change its identification. Why is it that on the musical domain it is still plausible to call the `ways of doing and making' in both regimes \emph{music}? Why within our culture someone who designs billboards is not considered to be a \emph{fine artist} (it probably would fall into graphic design) while someone who writes jingles for television commercials is still a \emph{musician}? Later, I will come back to this questions and look at the possible reasons and implications of this difference. However, before drawing any conclusions about the consequences of this disparity, first I will examine the \emph{aesthetic regime of art} to have a better understanding of Ranci\`{e}re's enquiry.

\subsubsection{The aesthetic regime of art and the shortcomings of the notion of modernity}

Ranci\`{e}re calls the \emph{aesthetic regime of art} that which liberates art from the \mbox{\emph{representative regime of art}} by breaking with its identification as the division of `ways of doing and making'. The \emph{aesthetic regime} therefore puts an end to the models used by the \emph{representative regime} and breaks the barriers of identification in the arts. It does so by distinguishing art as an occupation that establishes, questions and alters the concept of what art is, it's hierarchies, subject matter and genres. 
\begin{quote}
The aesthetic regime of the arts is the regime that strictly identifies art in the singular and frees it from any specific rule, from any hierarchy of the arts, subject matter, and genres. Yet it does so by destroying the mimetic barrier that distinguished `ways of doing and making' affiliated with art from other `ways of doing and making', a barrier that separated its rules from the order of social occupations. The aesthetic regime asserts the absolute singularity of art and, at the same time, destroys any pragmatic criterion for isolating this singularity.\footnote{Ibid., p. 23.}
\end{quote}
Hence, the \emph{aesthetic regime} establishes the autonomy of art and at the same time makes art independent of it's own forms. As a result, the artist becomes a practitioner of a discipline specific to whatever falls into the category of art. 

At this point, one should think of the \emph{aesthetic regime} in the domain of music. I will propose that music that falls within this regime is music that challenges the \emph{poetic regime} and the very notion of \emph{what music is} at a given point in time. It also should be thought as a regime that makes music independent from it's own subject matter, rules, conventions and genres, and frees it from specific `ways of doing and making'. It changes music's visibility and makes it autonomous from the very notion of itself, from it's expected `musical' and social functions.\footnote{Here, I refer to `social functions' not as in the purpose or use of music within the \emph{ethical regime}, but the social functions it performs within the \emph{poetic regime}.} In the history of music, it is easy to think of examples of music that breaks with musical practices of its time and redefines itself\footnote{There are too many examples for me to list them here.}. It is even possible to think of brief historical periods before the twentieth century where one can observe some form or manifestation of the \emph{aesthetic regime} in music. Nevertheless, it is difficult to think of music as an autonomous discipline, freed from it' own \emph{substance}. That is to say, even though the definition of music has changed and was challenged in several occasions, it was not until the twentieth century that the concept fully emerged of `the musician' as someone who creates whatever he conceders suitable music to be and is not expected to follow traditional formulas of music-making. Even to this day, I think that this concepts of music and `the musician' are not completely widespread within the community.\footnote{Later, I will propose several possible reasons for this problem.}

Ranci\`{e}re, goes further to examine the limitations of the notion of modernity and it's relationship to the \emph{aesthetic regime of art}. He describes what commonly is referred to as \emph{modernism} in art as an `incoherent' label designated instead of what truly should be attributed to the \emph{aesthetic regime of art}. There is a sort of simplicity ascribed to the notion of modernity that is viewed as a clear line of transition or rupture from the old to the new and in the case of the visual arts between figurative and non-figurative representation. Ranci\`{e}re argues that the break from figurative representation is a confusion that emerged from the simplistic view that this break would mean a rupture from the \emph{representative regime of art}.

\begin{quote}
The basis for this simplistic historical account was the transition to non-figurative representation in painting. This transition was theorized by being cursorily assimilated into artistic `modernity's' overall anti-mimetic destiny. . . . However, it is the starting point that is erroneous. The leap outside of \emph{mim\={e}sis} is by no means the refusal of figurative representation.\footnote{Ibid., p. 24.}
\end{quote}

Therefore, the break from figurative representation does not mean the establishment of a new visibility for art nor a break from the mimetic barrier of `ways of doing and making'. Moreover, Ranci\`{e}re asserts that the contradiction of the \emph{aesthetic regime of art} which on the one hand establishes the autonomy of art and on the other hand questions the distinction between art and other activities leads to two big misunderstandings of the notion of modernity. The first confusion was to simply associate the modernist movement with the autonomy of art. The modernist project was therefore reduced only to an `anti-mimetic' movement that concentrates on the idealistic concept of stripping away from all references of previous art forms and works in order to reveal art's `purity' of form and reach it's `essence'. They attempted this by exploring only the formal aspects of art by focusing on the capabilities of it's own medium. The second big confusion, according to Ranci\`{e}re, is the idea that the forms of the \emph{aesthetic regime of art} were somehow related to other forms that would materialize by accomplishing a task or fulfilling a destiny specific to modernity. In other words, the revolution that rendered autonomy to art became the example for the Marxist revolution. The failure of both the `anti-mimetic' principles of modernism and the political revolution resulted in a `crisis of art' caused by this paradigms of modernism. Modernism in art therefore ``became something like a fatal destiny based on a fundamental forgetting.''\footnote{Ibid., p. 27.} 

\subsection{Musical modernity: misconceptions and misunderstandings}

I will propose that a similar confusion has taken place in western music, which leads to analogous misunderstandings regarding the so called modernist project. However, in order to avoid simplifications, one should first remember certain aspects about the state of western music at the end of the nineteenth and beginning of the twentieth centuries. It is important first of all to realize that due to certain developments in western music by the end of the nineteenth century there was a clear specialization of musicians---some were trained specifically as performers and others as composers. This division of occupations in music lead to a greater dichotomy in the `ways of doing and making' music. The specificity of the performer's creative decisions therefore became mostly linked to the realization of a given score. The composer's role, on the other hand, was to provide a score to the performers and establish certain directions and instructions on parameters such as in pitch, rhythm, musical form and instrumentation to name a few. During this time, the role of the composer became more prominent concerning music innovation and therefore most of these developments are attributed mostly to composers in western music history. Hence, I will mostly refer to composers in attempting to explain the limitations of the notion of musical modernity. Nevertheless, by no means I am attempting to discredit or ignore the performers' role---I am just referring to the more widespread view of these developments. I will later explain in more detail the complexities of this division of occupations in wester music regarding the \emph{aesthetic regime} but in the meantime, in my analysis I will refer mostly to composer's work. To add to the complexity of this problem, I will also mention that it would be slightly reductionist of me to assume that composers at this time where fully aware of the transformation that music was undergoing. When one looks at the writings from this period, one can not clearly see an explicit intentionality to give autonomy to music, redefine its visibility or revolutionize its `ways of doing and making'. Nevertheless, I will argue that even though they were not fully aware of the implications of this transformations, some of their work and writing point towards the establishment of an \emph{aesthetic regime} in music.

At the end of the nineteenth century, composers such as Wagner, Mahler and Debussy, to name a few, were already expanding the tonal system through the so called `emancipation of dissonance,' signaling what was to become a radical break in music history---that is, Sch\"{o}nberg's moving away from the tonal system altogether and starting to compose freely, without following the tonal system. This radical gesture signals a step towards the \emph{aesthetic regime} in that this action attempts to free music from previous musical models---in this case it frees itself from the tonal system---thus venturing to unleash music from it's own \emph{substance}. Sch\"{o}nberg, with he's period of so called ``free atonality''\footnote{The period between 1908 and 1923 in which Sch\"{o}nberg abstained from using tonality and did not adhere to a systematic method of pitch organization.} and later with his twelve-tone method\footnote{Devised by Sch\"{o}nberg in 1921 and first described to his inner circle in 1923.}, breaks away from the convention that a composer should follow previous models of composition and starts to define a new notion of the composer as someone who decides what he considers music to be and chooses how it is to be organized. Therefore, the rupture from the tonal system at the beginning of the twentieth century challenges the definition of music in western society and contributes to redefine `the musician' as someone who does not follow existing models, but can invent his own modes and systems of music-making. However, it is important to note that the break from tonality by no means represents the establishment of an \emph{aesthetic regime} in music nor a leap outside \emph{mim\={e}sis} and the \emph{representative regime}.  Stravinski's \emph{Le Sacre du Printemps}\footnote{Premiered in Paris, 1913.}, is a clear example of a work that points towards the \emph{aesthetic regime} but does so not by abandoning tonality, but by breaking with other models of concert music. The radicality of \emph{Le Sacre du Printemps} comes from developments in musical parameters such as rhythm, tonality (polytonality, etc), timbre and musical form, but not from a complete renounce from tonality. Stravinsky's use of folk-music, primitive rhythms, asymmetric structures and orchestral textures was music never heard before and stretched the definition of concert music as well as proposed new ways of organizing it's subject matter, freeing music from specific `ways of doing and making'. At the same time Stravinsky invents new rules and defies traditional genres and styles, which are all characteristics of the \emph{aesthetic regime}. 

Sch\"{o}nberg's importance in the establishment of the \emph{aesthetic regime} is also not to be discredited and I believe that by departing from tonality, he certainly redefined \emph{what music is} and questioned music's subject matter. Moreover, through his revolutionary shock on the community's notion of music, he certainly contributed to changing the notion of `the musician' as someone who produces what \emph{he considers music to be}. It is also compelling to see that Sch\"{o}nberg's use of dissonance was not with the purpose of centering his musical discourse around pitch organization or being non-referential to previous musical styles and genres. Paradoxically, even though his way of organizing pitches was radically new, he was fairly traditional in his use of other musical parameters such as form\footnote{He constantly used traditional forms such as sonata form, suite and theme and variations.}, timbre and gesture. For Sch\"{o}nberg, the method by which he organized notes and used atonality were not very important elements in his work.

\begin{quote}
I personally do not find that atonality and dissonance are the outstanding features of my works. They certainly offer obstacles to the understanding of what is really my musical subject.\footnote{Arnold Sch\"{o}nberg, \emph{Style and Idea}, Trans. Leo Black, Los Angeles: University of California Press, 1984, p. 77.}
\end{quote}
This separates him from the next generations of composers that embraced his twelve-tone system and who's main compositional objective focused on the organization of this twelve pitches. It is by trying to understand this next generation of composers' work that Ranci\'{e}re's analysis of the confusion of the notion of modernity comes handy. It is crucial to remember the fist confusion, which is to simply seek the autonomy of art through `anti-mimetic' strategies. In the case of music, this was attempted by focusing on formal aspects of music such as how to organize pitches, rhythms, dynamics and all other possible `musical' parameters. By giving importance to the formal aspects of the compositional medium they sought to stretch music's capabilities and to seek music's autonomy by striping it away from all references of other musics. It is fascinating to read that when Sch\"{o}nberg showed his twelve-tone method to his associates in 1923, he already could notice the potential problems of looking at music only in terms of the formal techniques implemented to compose it. 
\begin{quote}
What I feared, happened. Although I had warned my friends and pupils to consider this as a change in compositional regards, and although I gave them the advice to consider it only as a means to fortify the logic, they started counting the tones and finding out the methods with which I used the rows. Only to explain understandably and thoroughly the idea, I had shown them a certain number of cases. But I refused to explain more of it, not the least because I had already forgotten it and had to find it myself. But principally because I thought it would not be useful to show technical matters which everybody had to find for himself and could do so. This is also the error of Mr. Hill. He also is counting tones and wants to know how I use them and whether I do it consequently.\footnote{Ibid., p.214.} 
\end{quote}

Sch\"{o}nberg's use of the twelve-tone method did not have an `anti-mimetic' purpose and he devised it to be able to have a systematic approach to musical form and to compose melodies, themes, phrases and chords. He also made clear his abandonment of the tonal system was not more important than other aspects of his work. It is important to note as well that after the invention of his method, he relied on gestures, orchestration and structures that where related to traditional styles and genres---specially those of the germanic tradition. Therefore, Sch\"{o}nberg's invention of the twelve-tone method was mostly pragmatic and did not have the purpose of not referring to other musics or focusing only in music's formal aspects. It is precisely these aspects of Sch\"{o}nberg's use of dodecaphony that later Boulez would criticize in his article ``Sch\"{o}nberg is dead.''
\begin{quote}
From Sch\"{o}nberg's pen flows a stream of infuriating clich\'{e}s and formidable stereotypes redolent of the most wearily ostentatious romanticism: all those endless anticipations with expressive accent on the harmony note, those fake appoggiaturas, those arpeggios, tremolandos, and note-repetitions, which sound so terribly empty and which so utterly deserve the label `secondary voices'; finally, the depressing poverty, even ugliness, of rhythms in which a few tricks of variation on classical formulae leave a disheartening impression of bonhomous futility.\footnote{Pierre Boulez, `Sch\"{o}nberg is dead', in \emph{Stocktakings from an Apprenticeship}, Oxford: Oxford University Press, 1991, pp. 212-213.}
\end{quote}
For what interested Boulez in the twelve-tone system were the formal aspects of the \emph{series}---an approach closer to Webern's dodecaphony. One can already see here in Boulez's position an `anti-mimetic' preoccupation to avoid clich\'{e}s and references to pervious traditional music as well as a modernist concern towards the formalization of music through the capabilities of serialism.

\begin{quote}
It has to be admitted that this ultra-thematicization is the underlying principle of the \emph{series}, which is no more than its logical outcome. Moreover, the confusion between theme and series in Sch\"{o}nberg's serial works is sufficiently expressive of his inability to envisage the world of sound brought into being by serialism. For him dodecaphony is nothing more than a rigorous means for controlling chromaticism; beyond its role as regulator, the serial phenomenon passed virtually unnoticed by Sch\"{o}nberg.\footnote{Ibid., p.212.}
\end{quote}

It was through the development of serialism in the fifties and sixties---lead by Boulez and Stockhausen---that composers would seek music's pure form through the serialization of all conceivable `musical' parameters, thus focusing only in an exploration of music only by it's own formal capabilities. The confusion caused by the establishment of the \emph{aesthic regime} that identifies modernity only with the autonomy of art and which lead to an `anti-mimetic' revolution became a major force in postwar european avant-garde. Serialism thus would seek through it's self-contained system an ideal of music that would avoid any external or `impure' elements and would attempt to escape any reference to other existing music. The scope of the serialist movement and it's influence over the avant-garde and `modernist' composers across the world should not be overlooked. 


%the notion of modernity is not to be overlooked. Composers of later generations across the world and even today, inherited from postwar composers `anti-mimentic' notions of music---they only look at music's formal aspects and abstract `musical' parameters.
%
%The 
%
%Cage - organizing sound - 
%spectralism (still avoids other musics to an extent - but uses acoustic phenomena) 


\subsection{The resurgence of representation in early postmodernism}

\newpage

\section{Technology, Appropiation and Postproduction} 

``Consumption is simultaneously also production, just as in nature the production of a plant involves the consumption of elemental forces and chemical material'' K. Marx

Sound Transformations: 					

``With the power of the computer, we can transform sounds in such radical ways that we can no longer assert that the goal sound is related to the source sound merely because we have derived one from the other.'' (T. Wishart)

In my work, sound transformations are used for the transformation of existing music. 

Why transformation of musical sources? Because they may carry complex cultural symbolism. 

The amount of processing can affect our ability to recognize the source sound or musical sample. Therefore, there is a wide palette of derivative music available to us: from the radically processed – less recognizable source – more `abstract' extreme; to the less processed – more recognizable source – more `referential'  and quotation type music.

Performance practice and other sonic characteristics of many original musical sources is lost in the transcription to a fully notated score for ensembles of western classically trained musicians. Many aspects of sound production (intonation, groove, spectral characteristics of instruments/voices, etc) is lost via this process.

Process of derivation and sound transformation is not directly apparent to the audience. The act of appropriation is not transparent.

Nicolas Bourriaud: Postproduction, 2002.

``Starting with the language imposed upon us (the system of production), we construct our own sentences (acts of everyday life), thereby reappropriating for ourselves, through these clandestine microbricolages, the last word in the productive chain''

``By listening to music or reading a book, we produce new material, we become producers. And each day we benefit from more ways in which to organize this production: remote controls, VCRs, computers, MP3s, tools that allow us to select, reconstruct, and edit. Postproduction artists are agents of this evolution, the specialized workers of cultural reappropriation.''

``Throughout the eighties, the democratization of computers and the appearance of sampling allowed for the emergence of a new cultural configuration, whose figures are the programmer and DJ. The remixer has become more important than the instrumentalist, the rave more exciting than the concert hall. The supremacy of cultures of appropriation and the reprocessing of forms calls for an ethics: to paraphrase Philippe Thomas, artworks belong to everyone. Contemporary art tends to abolish the ownership of forms, or in any case to shake up the old jurisprudence. Are we heading toward a culture that would do away with copyright in favor of a policy allowing free access to works, a sort of blueprint for a communism of forms?'' (N. Bourriaud)

\subsection{The postmodern condition and `the society of the spectacle'}

resurgence of image / music quotations/references - first as reaction to the anti-mimetic
later with digital technology, easy reproduction, etc, etc => the use of images becomes the same as before the establishment aestetic regime : commodification, capitalism, DJ culture, digital quotations (in hip-hop, sound libaries, etc, etc)

\subsection{critisisms} 
\subsection{The liberal-comunists: Open Source, etc.} 

There is no music by John Oswald on the net free to download. Hypocrisy from the appropriator? Or does he fall into the logic of late-capitalism - “no communism of forms”? “I plunder but don’t plunder me. Or, at least not for free…”? 

I propose an attitude towards music appropriation similar to that of hacker communities and the open source initiative. Not with the purpose of suggesting a communist utopia, but of being consequent with my creative process. By giving away my music, recorded sounds and experiments, code, etc, through the net, I will hopefully instigate others to do so as well. If this attitude is followed, it could promote the organization of music cyber communities that would plunder, engage with and promote each other, hopefully producing more subversive types of music.

We are far from the Bourriaud’s utopia. The only people how have access to (artistic) shareware are commoditized people, mostly in western countries. Isn’t the DJ approach towards plunderphonics one that appropriates to make more profit and diminish costs only to thereafter feed back their product into the music industry system?

 
\section{Radical Musics: Resurecting the modernist political project?}
The distribution of the sensible. Ranciere.
The Emancipated Spectacle.
Did Music break with the mimetic regime? Is it still expected for a musician or composer to do something? Micheal Hard - instead of coming with new concepts (`Art' Music, Sonic Arts, Sonology, etc), why don't we struggling with the old one?

\subsection{Institutions, Music Industry, etc} 

\section{Computer-Mediated Musicking}
Christopher Small argues that music is not a thing or an abstract concept, but a human activity that he calls \emph{musicking}, meaning all individual and collective endeavors in the process of music making. Moreover, Small questions the notion that a musical work is what gives meaning to \emph{musicking}. 

\begin{quote}
The act of \emph{musicking} establishes in the place where it is happening a set of relationships, and it is in those relationships that the meaning of the act lies. They are to be found not only between those organized sounds which are conventionally thought of as being the stuff of musical meaning but also between the people who are taking part . . . relashionships between person and person, between individual and society, between humanity and the natural world. . . . (Small, 1998)
\end{quote}

The music we compose and perform can convey our thoughts and express our feelings. As listeners we interpret . . . make us feel and think. Empathy. Exchange.  

\subsection{Compositional Stratergies based on reshaping relationships in music making}

Gilius piano pieces.

\subsection{Reshaping relationships in music making through technology?}

The introduction of electroacoustic resources into live musical performance has changed the relationship between the composer and the performer. 

The use of computer technology has also fostered new collaborative possibilities between performers of different cultures.

Musicians of different backgrounds (improvisation and notated music) and traditions (Western and non-Western) may now share the stage simultaneously and productively through technology; in spite of previously incompatible performance conventions.

Real-Time computer processing allows the possibility of using the audio signal (as well as other information - like MIDI) from several live performances simultaneously as building blocks for a composition.

\subsection{Evaluating Human and Machine Performance}
\subsubsection {Iteration}
\subsubsection {Generative Music + AI}

\section{Appropriation as a Compositional Strategy}

\subsection{Musica Derivata and Plunderphonics}

``A good composer does not imitate; he steals''       I. Stravinsky

Musica Derivata:

``music that is compositionally based on other music'' (K. Barlow) 

Plunderphonics:

John Owald, 1985. ``Plunderphonics, or Audio Piracy as Compositional Prerogative''

Use of audio samples as a technique for composition. 

Different from Musica Derivata in that it appropriates the recording of the original musical source. Information from recording (tibre, rhythm, performace practice, etc) is plundered from the original source to create a new composition.

``As a listener my own preference is the option to experiment. My listening system has a mixer instead of a receiver, an infinitely variable speed turntable, filters, reverse capability, and a pair of ears. An active listener might speed up a piece of music in order to perceive more clearly it’s macrostructure, or slow it down to hear articulation and detail more precisely'' (J. Oswald)

\subsection{Redefining the `Real' in Real-Time: A Lacanean reading of Live-Electronic Performance}
Much has been written about the problematics of live electronic music performance using computers.\footnote{ See Barrett(2008), Croft(2007), d'Escriv\'{a}n(2006) and Emmerson(2007)} Most of the discussion seems to be centered in how to define what `live' means in a performance using computer technology which escapes the ``well-understood Newtonean mechanics of action and reaction, motion, energy, friction and damping.''\footnote{ (Emmerson, 2007).} The problem of what appears to be \emph{real} regarding a computer performance is a continuing source of debate. There are some with the position that the relationship between physical action and sonic reaction must remain for a performance to continue to have `liveness' and meaning (Croft, 2007)\footnote{`Theses on liveness'. \emph{Organised Sound} 12(1), p. 59-66. 2007 Cambridge University Press.} while others argue that a new generation of musicians are satisfied with having no apparent correlation between physical effort and sound output (d'Escriv\'{a}n, 2006)\footnote{`To sing the body electric: Instruments and effort in the performance of electronic music', \emph{Contemporary Music Review}, Volume 25, Issue 1 and 2 February 2006, pages 183-191}. 
\begin{itemize}
\item
Barrett's position. 
\item
Simon Emmerson Real and Imaginary Relationships
\item
Lacanean `Real', `Imaginary' and `Simbolic'
\end{itemize}


\subsection{plunderphomes, ideology and the use of references}

Ranciere: 
\begin{quote}
And the time came when the semiologist discovered that the lost pleasure of images is too high a price to pay for the benefit of forever transforming mourning into knowledge. . . .
\end{quote}
\begin{quote}
While some start up a prolonged lamentation for the lost image, others reopen their albums to rediscover the pure enchantment of images- that is, the alterity of the \emph{was}, between the pleasure of pure presence and the bit of the absolute Other.
\end{quote}
\begin{quote}
Evidence of exhibitions devoter to `images', but also the dialectic that affects each type of image and mixes its legitimations and powers with those of the other tow.
\end{quote}
Plunderphones reflect ideology . . . \v{Z}i\v{z}ek/Adorno but. . . . The artist can present their own view of these references by rearranging them modifying them. The plunderphonics artist doesn't necessarily adheres to the ideology of the appropriated material, but reflects it by the use of the plunderphones - how are they presented, modified, etc?  

\subsection{On Appropriation}

What? 

Code, compositional tecniques, what piece of music? 
Do we plunder from the ``flea market or (the) airport shopping mall''? (N. Bourriaud). From the top 20 list - J. Oswald approach-, or from the hidden CDs at the back of the music store?

Who?

Music Industry? Pop/commercial? Historical (dead composers)? Music from different cultures? 

Appropriation of the Other. What relationship do we want to establish with the Other? Impersonal like the 1st/3rd World relationships?

Liberal multiculturalists approach? ``Other deprived of its Otherness (the idealized Other who dances fascinating dances and has an ecologically sound holistic approach to reality, while features like wife beating remain out of sight…)?'' (Slavoj \v{Z}i\v{z}ek, 2003)

Why?

For the meaning of the cultural object you are appropriating? For it’s symbolism? To suggest a metaphor?

For it’s use? ``Don’t look for the meaning, look for the use'' - L. Wittgenstein - for example for the sonic qualities of the appropriation (intonation, groove, etc.)

How? …

\subsection{Real-Time Plunderphonics}

Appropriation of audio signals from live music performances as material for a new composition

Creates a cognitive dissonance between audio and visuals.

The amount of processing of the audio signals is visible. The more processed the performances are, the more contrasting they will look in relationship with what is heard through speakers.

In contrast to acousmatic tradition, Real-Time Plunderphonics makes the process of appropriation transparent to the audience through the cognitive association between audio and visuals.

Changes relationship with the appropriated Other: The performer becomes an accomplice in the process of appropriation (or themselves). 

Deals with the problematic of the lack of visual clues and theatrical elements in electronic music performance by introducing a dynamic group of live performers and an interesting and unusual visual scenario.  

\subsubsection{Some ideas of how to plunder}

Get to know what and who you are plundering and figure why your are doing so before you decide how to plunder.(Know your performers, their music and why you want to work with them)

Appropriate and plunder yourself. 

Plundering not as central purpose of the creative process, but rather a tool for creating new idiosyncratic audio/visual result. 

Use ``from raw to cooked'' (L\'{e}vi-Strauss) techniques to create a narrative that navigates, in literary terms, between the ‘real’ (actual performance) and the `surreal' (extreme processed audio).

Combinations of Real-Time Plunderphonics, (Real-Time) Musica Derivata and Sound Transformations

Use plunderphones as data: reprogram, not just remix.

Micro and macro plundering.

Use also Non Real-Time tools (Scores, Samples, etc.) if suitable. 

Using plunderphones as data

An example: Use FFT data of your plunderphone to trigger samples of recorded instruments.

\subsubsection{Micro and Macro Plundering}

Microplunderphonics

Plundering just microelements of sound. Not the whole spectrum of the original sound file. 

Generate noise with your plunderphones and use it instead of white noise for sound synthesis


Macroplundering

Appropriate a composition’s form. Use the structure as blueprint for a new composition. 

Use variables of the appropriated piece (pitch, dynamics, etc.) as control structures for new output.

\subsection{Crossing Cultural Borders?}

A discussion of Simon Emmerson's Crossing Cultural Boundaries through Technology. 
\v{Z}i\v{z}ek's view of Multiculturalism. 


\subsection{Interpassivity}

\begin{quote}
Interpassivity, like interactivity, thus subverts the standard opposition between activity and passivity: if in interactivity (or the cunning of Reason), I am passive while being active through another, in interpassivity, I am active while being passive through another. More precisely, the term interactivity is currently used in two senses: (1) interacting with the medium, that is, not being just a passive consumer: (2) acting through another agent, so that my job is done, while I sit back and remain passive, just observing the game. While the opposite of the first mode of interactivity is also a kind of interpassivity, the mutual passivity of two subjects, like two lovers passively observing each other and merely enjoying each others presence, the proper notion of interpassivity aims at the reversal of the second meaning of interactivity: the distinguishing feature of interpassivity is that, in it, the subject is incessantly (frenetically even) active, while displacing on to another the fundamental passivity of his or her being.\footnote{From The Fantasy in Cyberspace by Slavoj \v{Z}i\v{z}ek}
\end{quote}



\label{ch:introduction}
