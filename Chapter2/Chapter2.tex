\chapter{Introduction}

%In this chapter, I will attempt to tackle different questions concerning the aesthetics surrounding my work with the purpose of introducing a conceptual framework that hopefully will situate it within a wider cultural and philosophical context. I will start by addressing some concerns regarding the notion of modernity within the musical discourse of western avant-garde composers and its approach towards tonality and making references to other musics. I will argue, through the work of french philosopher Jaques Ranci\`{e}re, that the departure from making references to other musics either from the western classical tradition or other ethnic backgrounds, was not part of the initial modernist project, but a misunderstanding of its original premises founded on the political desire of establishing what he calls the \emph{aesthetic regime} of the arts. Nevertheless, before dwelling into this debate, in order to fully understand this confusion and later grasp certain concepts that will be useful in understanding some of my artistic motivations, I will examine Ranci\`{e}re's idiosyncratic and revealing view on aesthetics and its relationship to politics.
%
%Later, I will continue by discussing the position of the first generation of composers of the so-called \emph{postmodernist} movement and their motivation to reintroduce references of other musics in their own work as a reaction to the anti-representative camp. Thereafter, I will continue by discussing certain aspects about music appropriation and the role of technology.

\section{Ranciere and the reevaluation of the \mbox{notion} of modernity}

Jaques Ranci\`{e}re in his book \emph{The Politics of Aesthetics} examines the relationship between the concept of modernity and the break from figurative representation in the visual arts. He argues that the departure from representation of images through figurative means is often confused with the modernist political notion that conceals a ``specific type of connection between ways of producing works of art or developing practices, forms of visibility that disclose them, and ways of conceptualizing the former and the later''.\footnote{Jaques Ranci\`{e}re, `The Distribution of the Sensible', in \emph{The Politics of Aesthetics}, Trans. Gabriel Rockhill, London: Continuum, 2004, p. 20.} If we were to think about this confusion that is associated with the concept of \emph{modernism} in the realm of music, some questions come into mind: Does this confusion apply to the musical domain and if so how does it manifest itself? Can we talk about representation in music and if so within what context? Could we compare the breaking from figurative representation to the departure from tonality at the beginning of the twentieth century? Has the musician gone through a corresponding redefinition of \emph{what is expected} from him by the community the same way as the `fine' artist has through the process of modernisation?

In the following discussion, I will attempt to read Ranci\`{e}re's text as applied to music not only with the purpose of tracing parallels between the other arts and music, but also to point out their differences and discrepancies regarding the limitations of the notion of modernity and its relationship to the wider modernist political project.

\subsection{The distribution of the sensible}

Before starting our discussion on the notion of modernity and its political and aesthetic consequences, first we will try to examine the relationship of aesthetics and politics in the work of Ranci\`{e}re. According to Ranci\`{e}re, the political and the aesthetic spheres are intrinsically linked through what he calls `The distribution of the sensible'. 
\begin{quote}
I call the distribution of the sensible the system of self-evident facts of sense perception that simultaneously discloses the existence of something in common and the delimitations that define the respective parts and positions within it. A distribution of the sensible therefore establishes at one and the same time something common that is shared and exclusive parts. This apportionment of parts and positions is base on a distribution of spaces, times, and forms of activity that determines the very manner in which something in common lends itself to participation and in what way various individuals have a part in this distribution.\footnote{Ibid., p. 12.}
\end{quote}
It is precisely this system of division of spaces, times and forms of activity that defines aesthetics and is also at the heart of politics. Therefore, aesthetics takes part in the political act of governing and in determining who the rulers are and how they come to power; as well as how the commons are distributed within a community. Here though, Ranci\`{e}re points out, that in order to make the relationship between politics and aesthetics, one must understand aesthetics ``in a Kantian sense -- re-examined perhaps by Foucault -- as the system of \emph{a priori} forms determining what presents itself to sense experience.''\footnote{Ibid., p. 13.} Aesthetics therefore should be seen here beyond the conventional view as strictly belonging to the confines of art and should not be seen merely as the `aesthetic practices' manifested in different artistic disciplines.  In contrast, in order to think of aesthetics in a context that could be applied outside of the arts, it requires its abstraction as modes of action, production, perception and thought; a system of ``delimitation of spaces and times, of the visible and the invisible, of speech and noise, that simultaneously determines the place and the stakes of politics as a form of experience.''\footnote{Ibid.}
Therefore, through the work of Ranci\`{e}re, it is possible to think of aesthetics in politics with a broader understanding of aesthetics as the distribution of the sensible. Moreover, for Ranci\`{e}re, `aesthetic practices' that disclose visibility in artistic practices reveal `ways of doing and making' that exist and have visibility within the community. There are different manifestations of this practices that confine an aesthetic distribution.

\begin{quote}
This forms define the way in which works of art or performances are `involved in politics', whatever may otherwise be the guiding intentions, artists' social modes of integration, or the manner in which artistic forms reflect social structures or movements... In this way, a sensible politicity exists that is immediately attributed to the major forms of aesthetic distribution such as theater, the page, or the chorus. There `politics' obey their own proper logic, and they offer their services in very different contexts and time periods.\footnote{Ibid., pp. 14-15.} 
\end{quote}
Consequently, it could be argued that there is an inherent political core in the way this artistic forms are constituted and within them lays a political project that renders the distribution of what is known to the community, the way it is organized and what remains visible or invisible.    

\subsection{The regimes of art}

In order to understand Ranci\`{e}re's reevaluation of the notion of modernity one must first understand what he calls the three `regimes of art', which are modes of identification and articulation between `ways of doing and making' and forms of visibility, as well as their conceptualization. In other words, the `regimes of art' simply distinguish different ways of making and thinking about `art' and how it is perceived.

\subsubsection{The ethical regime of images and the poetic regime of art}

To begin with, Ranci\`{e}re defines the \emph{ethical regime of images} as the pragmatic Platonic notion of the use and distribution of images in relationship to the community's \emph{ethos}. This regime therefore uses images as `true' imitations of the original and are distributed and valued by their purpose of educating the community in accordance to it's social order. Therefore, within this regime `art' is not evaluated by qualities within itself but by their purpose in the community. He goes on to define a \emph{poetic regime of art} (also referred to as \emph{representative regime of art}) as that which breaks away from the \emph{ethical regime of images} and values the arts in terms of their own \emph{substance}.

\begin{quote}
I call this regime \emph{poetic} in the sense that it identifies the arts -- what the Classical Age would later call the `fine arts' -- within a classification of ways of doing and making, and it consequently defines proper ways of doing and making as well as means of assessing imitations. I call it \emph{representative} insofar as it is the notion of representation or \emph{mim\={e}sis} that organizes these ways of doing, making, seeing and judging. Once again, however, \emph{mim\={e}sis} is not the law that brings the arts under the yoke of resemblance. It is first of all a fold in the distribution of ways of doing and making as well as in social occupations, a fold that renders the arts visible. It is not an artistic process but a regime of visibility regarding the arts.\footnote{Ibid., p. 22.}
\end{quote}

Here, we come to the first deadlock in reading Ranci\`{e}re's notion of the `regimes of art' as applied to music. How are we to understand the difference between the \emph{ethical regime of images} and the \emph{representative regime of art} outside the domain of the visual and fine arts? First, we should understand that music not only has different social functions and visibility, but within it's unique organization it has particular `ways of doing and making' that are specific to itself. Even though music occupies a different and particular position in the ways of distributing the sensible, I will continue to argue that we can still refer to the \emph{ethical} and the \emph{poetic} regimes in music. 

Following Ranci\`{e}re categorization, I will refer to music within the \emph{ethical regime} as music that is made, heard and judged for it's purpose within the community. By this, I mean music that is not assessed by it own qualities -- or as Ranci\`{e}re would say `by it's own \emph{substance}' -- but by the purpose it performs within the community. Examples of this in the western tradition would include church, court and military music to mention just a few. It is easy to find music that falls within the \emph{ethical regime} in other cultures where in some cases music is not even differentiated from other disciplines, like dance or storytelling, and is performed (in some cultures everyone partakes in music-making) and valued by members of the group by it's communal and ceremonial purposes (celebration, mourning, war, etc). Of course, we can still find many examples of the \emph{ethical regime} today in music for theater, dance, television, films and religious purposes. I am also of the opinion that a large quantity of Pop Music is also not appreciated by it's own attributes but is judged on the bases of making profit within a consumer society (selling records, tickets and other merchandize). That said, I want to make clear that I am not attempting to devalorize or make a value judgment about music that falls within the \emph{ethical regime}. Furthermore, some music might also qualify within more than one regime simultaneously.

I will define music that falls within the \emph{poetic regime} as that which is appreciated for its own \emph{substance} but still follows or imitates a model.\footnote{By model I not only mean the written but also the unwritten rules in music performance and composition. The written rules could be for example tretises of harmony and orchestration whereas the unwritten rules could be performance practices and conventions in composition and improvisation, too name a few.} Namely, music that is judged by it's own `musical' qualities, and that is made with the main purpose of been listened to and evaluated by it's own subject matter. This music would be \emph{representative} insofar as it imitates or resembles a musical model (for example rules of harmony, counterpoint or sonata form, to mention just a few). A lot of western `concert music' would follow in this category in that it is made, heard and valued for it's `musical' qualities and judged as good or bad, adequate or inadequate, satisfactory or not, based on how the performer or composer follows certain models -- in the case of the performer, models of performance practice, and in the case of the composer, compositional models such as chord progressions, voice-leading, musical themes, variations, etc. 

It is interesting to note that even though in the visual arts the breaking from the \emph{ethical regime of images} and the establishment of the \emph{poetic regime of art} is what separates the `fine arts' from other modes and techniques of production (of images, shapes, objects, etc), in music there is not such a change in definition. That is to say, in the visual arts this break between \emph{ethical} and \emph{poetic} regimes identifies the arts as such but in music it does not change its identification. Why is it that on the sonic domain we still call the `ways of doing and making' in both regimes \emph{music}? Why within our culture someone who designs billboards is not considered to be a fine artist (it probably would fall into graphic design) while we still call someone who writes television jingles a musician? Later, we will come back to this questions and look at the possible reasons and implications of this difference. However, before drawing any conclusions about the consequences of this disparity, first we will examine the \emph{aesthetic regime of art} to have a better understanding of Ranci\`{e}re's enquiry.

\subsubsection{The aesthetic regime of art and the shortcomings of the notion of modernity}

Ranci\`{e}re calls the \emph{aesthetic regime of art} that which liberates art from the \mbox{\emph{representative regime of art}} by breaking with its identification as the division of `ways of doing and making'. The \emph{aesthetic regime} therefore puts an end to the models used by the \emph{representative regime} and breaks the barriers of identification in the arts. It does so by distinguishing art as an occupation that establishes, questions and alters the concept of what art is, it's hierarchies, subject matter and genres. 
\begin{quote}
The aesthetic regime of the arts is the regime that strictly identifies art in the singular and frees it from any specific rule, from any hierarchy of the arts, subject matter, and genres. Yet it does so by destroying the mimetic barrier that distinguished ways of doing and making affiliated with art from other ways of doing and making, a barrier that separated its rules from the order of social occupations. The aesthetic regime asserts the absolute singularity of art and, at the same time, destroys any pragmatic criterion for isolating this singularity.\footnote{Ibid., p. 23.}
\end{quote}
Hence, the \emph{aesthetic regime} establishes the autonomy of art and at the same time makes art independent of it's own forms. As a result, the artist becomes a practitioner of a discipline specific to whatever falls into the category of art. 

At this point, we should think of the \emph{aesthetic regime} in the domain of music. I will propose that music that falls within this regime is music that challenges the \emph{poetic regime} and the very notion of \emph{what music is} at a given point in time. It also should be thought as a regime that makes music independent from it's own subject matter, rules, conventions and genres, and frees it from specific ways of doing and making. It changes music's visibility and makes it autonomous from the very notion of itself, from it's expected `musical' and social functions.\footnote{Here I refer to `social functions' not as in the purpose or use of music within the \emph{ethical regime}, but the social functions it performs within the \emph{representative regime}.} In the history of music, it is easy to think of examples of music that breaks with musical practices of that time and redefines itself\footnote{There are too many examples for me to list them here.}. Nevertheless, it is difficult to think of music, before the twentieth century, as an autonomous discipline, freed from it' own \emph{substance}. That is to say, even though the definition of music changed and was challenged in several occasions, it was not until the twentieth century that the concept emerged of the musician as someone who creates whatever he conceders suitable music to be, and is not expected to follow traditional formulas and well-known models of music-making. Even to this day, I think that this concept of music and `the musician' is not very widespread within the community.  

Ranci\`{e}re, goes further to examine the limitations of the notion of modernity and it's relationship to the \emph{aesthetic regime of art}. He describes what we commonly refer to as \emph{modernism} in art as an `incoherent' label designated instead of what truly should be attributed to the \emph{aesthetic regime of art}. There is a sort of simplicity ascribed to the notion of modernity that is viewed as a clear line of transition or rupture from the old to the new and in the case of the visual arts between figurative and non-figurative representation. Ranci\`{e}re argues that the break from figurative representation is a confusion that emerged from the simplistic view that this break would mean a rupture from the \emph{representative regime of art}.

\begin{quote}
The basis for this simplistic historical account was the transition to non-figurative representation in painting. This transition was theorized by being cursorily assimilated into artistic `modernity's' overall anti-mimetic destiny... However, it is the starting point that is erroneous. The leap outside of \emph{mim\={e}sis} is by no means the refusal of figurative representation.\footnote{Ibid., p. 24.}
\end{quote}

Therefore, the break from figurative representation does not mean the establishment of a new visibility for art nor a break from the mimetic barrier of ways of doing and making. Moreover, Ranci\`{e}re asserts that the contradiction of the \emph{aesthetic regime of art} which on the one hand establishes the autonomy of art and on the other hand questions the distinction between art and other activities leads to two big misunderstandings of the notion of modernity. The first confusion was to simply associate the modernist movement with the autonomy of art. The modernist project was therefore reduced only to an `anti-mimetic' movement that concentrates on the idealistic concept of stripping away from all references of previous art forms and works in order to reveal art's `purity' of form and reach it's `essence'. They attempted this by exploring only the formal aspects of art by focusing on the capabilities of it's own medium. The second big confusion, according to Ranci\`{e}re, is the idea that the forms of the \emph{aesthetic regime of art} were somehow related to other forms that would materialize by accomplishing a task or fulfilling a destiny specific to modernity. In other words, the revolution that rendered autonomy to art became the example for the Marxist revolution. The failure of both the `anti-mimetic' principles of modernism and the political revolution resulted in a `crisis of art' caused by this paradigms of modernism. Modernism in art therefore ``became something like a fatal destiny based on a fundamental forgetting.''\footnote{Ibid., p. 27.} 


At this point, I will propose that a similar confusion has taken place in western music, which  leads to analogous misunderstandings regarding the so called modernist project. However, in order to avoid simplifications we should first remember certain aspects about the state of western music at the end of the nineteenth and beginning of the twentieth centuries. It is important first of all to realize that due to certain developments in western music by the end of the nineteenth century there was a clear specialization of musicians and some were trained specifically as performers and others as composers. This division of occupations in music lead to a greater dichotomy in the `ways of doing and making' music. The specificity of the performer's creative decisions therefore became mostly linked to the realization of a given score. The composer's role, on the other hand, was to provide a score to the performers and establish certain directions and instructions on parameters such as in pitch, rhythm, musical form and instrumentation to name a few. During this time, the role of the composer became more prominent concerning music innovation and therefore most of these developments are attributed mostly to composers in western music history. Hence, I will mostly refer to composers in attempting to explain the limitations of the notion of musical modernity. Nevertheless, by no means I am attempting to discredit or ignore the performers' role; I am just referring to the more widespread view of these developments. I will later explain in more detail the complexities of this division of occupations in wester music regarding the \emph{aesthetic regime} but in the meantime, in my analysis I will refer mostly to composer's work. To add to the complexity of this problem, I will mention that it would be a slightly reductionist point of view to assume that the composers at this time where fully aware of the transformation that music was undergoing. When we look at the writings from that period, we can not clearly see an explicit intentionality to give autonomy to music, redefine its visibility or revolutionize its `ways of doing and making'. Nevertheless, I will argue that even though they were not fully aware of the implications of this transformations, some of their work and writing points towards the establishment of an \emph{aesthetic regime} in music.

Emancipation of dissonance (Wagner, Debussy, Mahler, etc) => Sch\"{o}nberg's free atonality and later twelve tone technique... signaling towards the aesthetic regime. The composer's attempt to free himself from musical substance and models (in this case through the rules of harmony) 

Examples of composers that start to define the \emph{aesthetic regime} in western music:

Stravinski's Rite of Spring => the composer decides and redefines what concert music should be (introducing folk-music, primitive rhythms, etc)

Sch\"{o}nberg's atonality and twelve tone method => the composer decides what he considers music to be and decides how it is to be organized, etc...

---
"pure music", anti-mimetic notions: Misunderstnings started with the interpretation of Schonberg's twelve-tone system: 

Sch\"{o}nberg about his twelve tone method:

\begin{quote}
What I feared, happened. Although I had warned my friends and pupils to consider this as a change in compositional regards, and although I gave them the advice to consider it only as a means to fortify the logic, they started counting the tones and finding out methods with which I used the rows. Only to explain understandably and thoroughly the idea, I had shown the a certain number of cases. But I refused to explain more of it, not the least because I had already forgotten it and had to find it myself. But principally because I thought it would not be useful to show technical matters which everybody had to find for himself and could do so. This is also the error of Mr. Hill. He also is counting tones and wants to know how I use them and whether I do it consequently (Sch\'{o}nberg, 1975, p.214). 
\end{quote}

Sch\"{o}nberg about the importance of atonality in his work:

\begin{quote}
I personally do not find that atonality and dissonance are the outstanding features of my works. They certainly offer obstacles to the understanding of what is really my musical subject.
\end{quote}

But here comes the first misunderstanding... 


\section{The resurgence of representation and musical references}

\subsection{Technology, Appropiation and Postproduction} 

\subsection{The postmodern condition and `the society of the spectacle'}

resurgence of image / music quotations/references - first as reaction to the anti-mimetic
later with digital technology, easy reproduction, etc, etc => the use of images becomes the same as before the establishment aestetic regime : commodification, capitalism, DJ culture, digital quotations (in hip-hop, sound libaries, etc, etc)

\subsection{critisisms} 
\subsection{The liberal-comunists: Open Source, etc.} 
 
\section{Radical Musics: Resurecting the modernist political project?}
The distribution of the sensible. Ranciere.
The Emancipated Spectacle.
Did Music break with the mimetic regime? Is it still expected for a musician or composer to do something? Micheal Hard - instead of coming with new concepts (`Art' Music, Sonic Arts, Sonology, etc), why don't we struggling with the old one?

\subsection{Institutions, Music Industry, etc} 

\section{Computer-Mediated Musicking}
Christopher Small argues that music is not a thing or an abstract concept, but a human activity that he calls \emph{musicking}, meaning all individual and collective endeavors in the process of music making. Moreover, Small questions the notion that a musical work is what gives meaning to \emph{musicking}. 

\begin{quote}
The act of \emph{musicking} establishes in the place where it is happening a set of relationships, and it is in those relationships that the meaning of the act lies. They are to be found not only between those organized sounds which are conventionally thought of as being the stuff of musical meaning but also between the people who are taking part.. relashionships between person and person, between individual and society, between humanity and the natural world... (Small, 1998)
\end{quote}

The music we compose and perform can convey our thoughts and express our feelings. As listeners we interpret... make us feel and think. Empathy. Exchange.  

\subsection{Compositional Stratergies based on reshaping relationships in music making}

Gilius piano pieces.

\subsection{Reshaping relationships in music making through technology?}

\subsection{Evaluating Human and Machine Performance}
\subsubsection {Iteration}
\subsubsection {Generative Music + AI}

\section{Real-Time Plunderphonics}

\subsection{Musica Derivata and Plunderphonics}

\subsection{Redefining the `Real' in Real-Time: A Lacanean reading of Live-Electronic Performance}
Much has been written about the problematics of live electronic music performance using computers.\footnote{ See Barrett(2008), Croft(2007), d'Escriv\'{a}n(2006) and Emmerson(2007)} Most of the discussion seems to be centered in how to define what `live' means in a performance using computer technology which escapes the ``well-understood Newtonean mechanics of action and reaction, motion, energy, friction and damping'' (Emmerson, 2007). The problem of what appears to be \emph{real} regarding a computer performance is a continuing source of debate. There are some with the position that the relationship between physical action and sonic reaction must remain for a performance to continue to have `liveness' and meaning (Croft, 2007)\footnote{`Theses on liveness'. \emph{Organised Sound} 12(1), p. 59-66. 2007 Cambridge University Press.} while others argue that a new generation of musicians are satisfied with having no apparent correlation between physical effort and sound output (d'Escriv\'{a}n, 2006)\footnote{`To sing the body electric: Instruments and effort in the performance of electronic music', \emph{Contemporary Music Review}, Volume 25, Issue 1 and 2 February 2006, pages 183-191}. 
\begin{itemize}
\item
Barrett's position. 
\item
Simon Emmerson Real and Imaginary Relationships
\item
Lacanean `Real', `Imaginary' and `Simbolic'
\end{itemize}


\subsection{plunderphomes, ideology and the use of references}

Ranciere: 
\begin{quote}
And the time came when the semiologist discovered that the lost pleasure of images is too high a price to pay for the benefit of forever transforming mourning into knowledge... 
\end{quote}
\begin{quote}
While some start up a prolonged lamentation for the lost image, others reopen their albums to rediscover the pure enchantment of images- that is, the alterity of the \emph{was}, between the pleasure of pure presence and the bit of the absolute Other.
\end{quote}
\begin{quote}
Evidence of exhibitions devoter to `images', but also the dialectic that affects each type of image and mixes its legitimations and powers with those of the other tow.
\end{quote}
Plunderphones reflect ideology ... Zizek/Adorno but... The artist can present their own view of these references by rearranging them modifying them. The plunderphonics artist doesn't necessarily adheres to the ideology of the appropriated material, but reflects it by the use of the plunderphones - how are they presented, modified, etc?  

\subsection{Crossing Cultural Borders?}

A discussion of Simon Emmerson's Crossing Cultural Boundaries through Technology. 
\v{Z}i\v{z}ek's view of Multiculturalism. 

\subsection{Micro and Macro Plundering}

\subsection{Interpassivity}


\label{ch:introduction}
